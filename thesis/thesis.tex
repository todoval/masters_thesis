%%% The main file. It contains definitions of basic parameters and includes all other parts.

%% Settings for single-side (simplex) printing
% Margins: left 40mm, right 25mm, top and bottom 25mm
% (but beware, LaTeX adds 1in implicitly)
\documentclass[12pt,a4paper]{report}
\setlength\textwidth{145mm}
\setlength\textheight{247mm}
\setlength\oddsidemargin{15mm}
\setlength\evensidemargin{15mm}
\setlength\topmargin{0mm}
\setlength\headsep{0mm}
\setlength\headheight{0mm}
% \openright makes the following text appear on a right-hand page
\let\openright=\clearpage

%% Settings for two-sided (duplex) printing
% \documentclass[12pt,a4paper,twoside,openright]{report}
% \setlength\textwidth{145mm}
% \setlength\textheight{247mm}
% \setlength\oddsidemargin{14.2mm}
% \setlength\evensidemargin{0mm}
% \setlength\topmargin{0mm}
% \setlength\headsep{0mm}
% \setlength\headheight{0mm}
% \let\openright=\cleardoublepage

%% Generate PDF/A-2u
\usepackage[a-2u]{pdfx}

%% Character encoding: usually latin2, cp1250 or utf8:
\usepackage[utf8]{inputenc}

%% Prefer Latin Modern fonts
\usepackage{lmodern}

%% Further useful packages (included in most LaTeX distributions)
\usepackage{amsmath}        % extensions for typesetting of math
\usepackage{amsfonts}       % math fonts
\usepackage{amsthm}         % theorems, definitions, etc.
\usepackage{bbding}         % various symbols (squares, asterisks, scissors, ...)
\usepackage{bm}             % boldface symbols (\bm)
\usepackage{graphicx}       % embedding of pictures
\usepackage{fancyvrb}       % improved verbatim environment
\usepackage[square,numbers]{natbib}         % citation style AUTHOR (YEAR), or AUTHOR [NUMBER]
\usepackage[nottoc]{tocbibind} % makes sure that bibliography and the lists
			    % of figures/tables are included in the table
			    % of contents
\usepackage{dcolumn}        % improved alignment of table columns
\usepackage{booktabs}       % improved horizontal lines in tables
\usepackage{paralist}       % improved enumerate and itemize
\usepackage{xcolor}         % typesetting in color
\usepackage{subcaption}    
\usepackage{afterpage}
\usepackage{makecell}
\usepackage{cleveref}
\usepackage{booktabs}
\usepackage{multirow}
\usepackage{enumitem}
\usepackage{listings}
\usepackage{algorithm}
\usepackage[noend]{algpseudocode}
\newcommand\mycommfont[1]{\footnotesize\ttfamily{#1}}
\usepackage{enumitem}
\newlist{steps}{enumerate}{1}
\setlist[steps, 1]{label = Step \arabic*}

%%% Basic information on the thesis

% Thesis title in English (exactly as in the formal assignment)
\def\ThesisTitle{Constrained Spectral Uplifting}

% Author of the thesis
\def\ThesisAuthor{Bc. Lucia Tódová}

% Year when the thesis is submitted
\def\YearSubmitted{2021}

% Name of the department or institute, where the work was officially assigned
% (according to the Organizational Structure of MFF UK in English,
% or a full name of a department outside MFF)
\def\Department{Department of Software and Computer Science Education}

% Is it a department (katedra), or an institute (ústav)?
\def\DeptType{Department}

% Thesis supervisor: name, surname and titles
\def\Supervisor{doc. Alexander Wilkie, Dr.}

% Supervisor's department (again according to Organizational structure of MFF)
\def\SupervisorsDepartment{Department of Software and Computer Science Education}

% Study programme and specialization
\def\StudyProgramme{Computer Science}
\def\StudyBranch{Computer Graphics and Game Development}

% An optional dedication: you can thank whomever you wish (your supervisor,
% consultant, a person who lent the software, etc.)
\def\Dedication{%
I would like to express my sincere gratitude to my supervisor, doc. Alexander Wilkie, Dr., for his time, guidance and patience over the last year.
I would also like to thank my boyfriend, my family and my friends for their support and encouragement.
}

% Abstract (recommended length around 80-200 words; this is not a copy of your thesis assignment!)
\def\Abstract{%
Physically-based spectral rendering is becoming increasingly popular in both commercial and academic areas due to its ability to accurately simulate natural phenomena. However, the production of materials defined by their spectral properties is a tedious and expensive process, which makes the utilization of RGB-based assets in spectral renderers a desired feature. To convert RGB values to their spectral representations, a process called spectral uplifting is employed. As the RGB color space is a finite subset of the visible gamut, there exist multiple conversion techniques producing distinct results, which may cause color inconsistencies under various lighting conditions. This thesis proposes a method for constraining the spectral uplifting process. To be specific, pre-defined mappings of RGB values to their spectral representations are preserved and the rest of the RGB gamut is plausibly uplifted. In order to assess its correctness, this technique is then implemented and evaluated in a spectral renderer. The renders uplifted via our method show minimal discrepancies when compared to the original textures.
}

% 3 to 5 keywords (recommended), each enclosed in curly braces
\def\Keywords{%
{spectral uplifting} {spectral rendering}
}

%% The hyperref package for clickable links in PDF and also for storing
%% metadata to PDF (including the table of contents).
%% Most settings are pre-set by the pdfx package.
\hypersetup{unicode}
\hypersetup{breaklinks=true}

% Definitions of macros (see description inside)
\include{macros}

% Title page and various mandatory informational pages
\begin{document}
\include{title}

%%% A page with automatically generated table of contents of the master thesis

\tableofcontents

%%% Each chapter is kept in a separate file
\chapter*{Introduction}
\addcontentsline{toc}{chapter}{Introduction}

Over the last few years, the demand for physical accuracy of the rendering process has grown substantially. By providing the renderer with the capability to physically simulate light transport, we can recreate natural phenomena such as metamerism, fluorescence, polarization, etc\ldots~To do so, the internal color representation is required to be as close to the visible light's spectral power distribution as possible. However, the advantages of this approach are often not worth the complexity of its implementation.

Even though the most correct physically-based approach to color representation is representing it as it is in nature, which is a spectral distribution of wavelength, its advantages are, in many cases, not worth the complexity of its implementation.

Therefore, vast majority of the conventional renderers internally represent color as an RGB tristimulus value, mainly due to the simplicity and the robustness of such systems. Additionally, almost all of the already existing assets, such as input textures and materials, are defined in RGB, as their creation and usage is fairly simple. Spectral assets require a real-life model whose reflectance must be measured with a spectrometer. Such a process is both tedious and, in many cases, even impossible.

To preserve the physical correctness of the spectral rendering process while utilizing RGB textures as its input, a conversion from the color's RGB representation to its spectral variant is needed. Such process is called \emph{spectral uplifting}, also known as \emph{spectral upsampling}.

However, the conversion process is not straightforward. As the RGB color space is intrinsically smaller than the gamut of visible light, there exist multiple (infinitely many) spectral representations of the same RGB color. Such spectral variations are called \emph{metamers}, and can cause inconsistencies under different lighting conditions. Various uplifting techniques might therefore provide different results, none of which necessarily have to match the real measured spectra.

The goal of this thesis is to extend an already existing uplifting system with the capability to constrain itself with pre-defined spectra:RGB mappings, which must be preserved during the uplifting process. All the other RGB input values should return plausible synthetic data which smoothly interpolates the measured spectra.
 
\section*{Related work}

Currently, there exist several approaches to spectral uplifting. They differ in the type of spectra they are capable of reconstructing, the type of gamut they can uplift, the color error caused by round trips, etc$\ldots$

Unfortunately, many of them have issues. The technique by~\citet{upsamplingMacAdam} is capable of creating only blocky spectra, which are unsuitable for smooth reflectances usually found in nature. The proposal by~\citet{upsamplingSmits}, although more widely used, is prone to slight round-trip errors, which arise from out-of-range spectra. One of the first approaches that produced smooth spectra was proposed by~\citet{upsamplingMeng}. However, they did not take energy conservation into account, which resulted in colors with no real physical counterpart, i.e. no real material could produce such color. \citet{upsamplingOtsu} introduce a technique that is capable of outperforming most of the existing approaches under specific conditions. Its drawback is its inability to satisfy the spectral range restrictions, which again causes color errors upon round trips.

In this thesis, we focus mainly on the technique proposed by~\citet{upsamplingJakobHanika}, which employs a pre-built model that is based on spectral representation using sigmoids. This algorithm produces smooth spectra satisfying spectral range restrictions with negligible error.

\citet{upsamplingFluorescence} further improve this technique for wide gamut spectral uplifting by introducing new parameters for fluorescence. Currently, this approach can be considered as state of the art.

None of the existing techniques propose a way in which to constrain the uplifting process.

\section*{Layout of this thesis}

This thesis is structured as follows:

In~\Cref{chap:colorScience}, we introduce the reader to light transport and color science. We explain the basic principles of human color perception, overview the existing color representations and focus on their importance in rendering.

\Cref{chap:spectralUplifting} summarizes the already existing spectral uplifting algorithms, placing special emphasis on the technique by~\citet{upsamplingJakobHanika}. Furthermore, it discusses the theory behind representation of spectra using moments.

\Cref{chap:implementation} details the implementation of our extension to an already existing spectral uplifting model and its utilization in a rendering software. It often refers to~\cref{chap:results}, where, in addition to discussing our results and comparing them to the non-constrained spectral uplifting, we provide multiple tests and experiments that assess the correctness of various possible implementations.
\chapter{Color Science}

Color science, or colorimetry, concerns itself with human perception of color. It researches the relations between human vision and physical properties of color, and analyzes options for both its capturing and reconstruction.

We begin this chapter by describing the physical properties of light and their subsequent meaning in terms of color. We then provide multiple options for quantifying said color for further possible reconstruction in the digital world(?). Lastly, we show the importance of color representation in modern-day renderers, such as Mitsuba or Corona (add a link).

\section{Light and Color}

The core of human visual perception is electromagnetic radiation, which consists of waves that propagate through space and transmit radiant energy.

An \emph{electromagnetic wave} is characterized by its \emph{amplitude} and \emph{frequency}. Amplitude is defined as the distance between the central axis and either the \emph{crest} (the highest point of the wave) or the \emph{trough} (the lowest point of the wave), while frequency specifies how many wave cycles happen in a second. Together, these properties give rise to the term \emph{wavelength}, denoted $\lambda$, which measures the length of the wave --- the distance between either two subsequent crests, troughs or any two following spots with the same height. 

Every electromagnetic wave can be unambiguously defined by its wavelength. Arranging them according to this criterion creates a classification known as \emph{electromagnetic spectrum} (see~\cref{fig:electromagneticSpectrum}). As the electromagnetic spectrum contains all existing types of electromagnetic radiation, it covers wavelengths in the range from fractions of nanometers to thousands of kilometers. This range is divided into bands to distinguish known types of light; low frequency light such as gamma rays or X-rays; extremely high frequency light such as radio waves.

\begin{figure}[t]
	\centering
	\includegraphics[width=0.8\linewidth]{img/electromagnetic_spectrum.png}
	\caption{An illustration of the electromagnetic spectrum~\cite{electromagneticSpectrum}} \label{fig:electromagneticSpectrum}
\end{figure}

In this thesis, we will focus on \emph{visible light}, which covers only a mere fraction of the electromagnetic spectrum. Its waves are roughly in the 380-780nm range.

To sum up, electromagnetic waves specify the way in which light travels. To, however, describe the interaction between light and matter, the term \emph{photon} is used. Photons are elementary particles of light moving in a manner specified by their wavelengths, making up electromagnetic radiation. They can be emitted or absorbed by atoms and molecules. During this process, they transfer energy either from the object that emitted them or to the object that absorbed them. This change in energy (denoted $E$) is proportional to the frequency of the absorbed/emitted photon and can be computed as follows~\citep{planckConstant}: 
\begin{equation} \label{energyEquation}
E = hf = \dfrac{hc}{\lambda}
\end{equation}
where $h$ is Planck's constant, $f$ is the frequency and $c$ is the speed of light. Therefore, generally speaking, the human eye identifies light when atoms and molecules in the retina absorb photons. 

To specify this process, we will first describe the retina. The retina consists of millions of light-sensitive cells, also called \emph{photoreceptors}, which pass a visual signal via an optic nerve to the brain, giving the notion of light and color. There are two types of photoreceptors in the human eye --- rods and cones.

\emph{Rods} make up most of the receptor cells (around 91 million according to~\citet{rods91cones4f5}, but other sources state that their number could be as high as 125 million~\cite{rods125cones6}). They are usually located around the boundary of the retina, and are responsible for low light (scotopic) vision. However, they possess very little notion of color, which is also the reason why the human eye has trouble recognizing colors during the night.

\emph{Cones} are located mainly in the center of the retina and their numbers are a lot lower (from around 4.5 million~\cite{rods91cones4f5} to 6 million~\cite{rods125cones6}). In contrast to rods, they are active at daylight levels (responsible for photopic vision) and have the notion of color. To be specific, different types of cones differ in their sensitivity to photon energies at concrete wavelengths. The final color is then composed by the brain from the stimulation signals sent by each cone.

The human eye has three types of cones:
\begin{itemize}
	\item \emph{L-cones}, which are the most responsive to longer wavelengths at around 560nm. When they are stimulated, they correspond to the red color.
	\item \emph{M-cones}, which are the most sensitive to medium wavelengths at around 530nm and correspond to green color
	\item \emph{S-cones}, which respond the most to small wavelengths that peak at around 420nm and correspond to blue color
\end{itemize}

Their relative response to stimulation can be seen in~\cref{fig:coneSensitivity}.
\begin{figure}[t]
	\centering
	\includegraphics[width=0.5\linewidth]{img/cone_sensitivity.png}
	\caption{Relative sensitivity of S, M and L-cones plotted according to the data measured by~\citet{coneSensitivities}.} \label{fig:coneSensitivity}
\end{figure}

This type of color perception is called \emph{trichromatic}, as it uses three types of receptors to create the whole color space.

The idea behind using three base colors has been adapted in color science to create multiple tristimulus color representations. We will discuss these more thoroughly in the following section.

Up until now, we have been talking about the interaction of light with the human eye. Photons, however, also interact with objects. As established by the relationship defined in~\cref{energyEquation}, the energy transferred to an object upon light interaction is dependent on the photon wavelength. This means that objects might absorb some wavelengths and reflect others.

Object color is defined by the wavelengths it \emph{reflects}. For example, if it reflects all the wavelengths, the resulting color is white, while absorbing all the wavelengths would render the object black. Naturally, human perception of object color is not only dependent on its reflective properties, but also on the lighting of the scene. If the only light present in the scene is red, other wavelengths than red will never hit the object. Therefore, the object might reflect only a subset of wavelengths than it would under white light, which might change the resulting color.

\section{Color representation}

The question of how to discretely represent color has been posed ever since the introduction of the first graphical user interface. For use in computer science, representations are required to be compact, precise, and the operations on colors should be easily executed.

We have already briefly mentioned the tristimulus representation in the previous section. In this section, we will overview its basic properties and describe some of the most popular tristimulus systems. We will also talk about an alternative representation, based primarily on the physical properties of color --- spectral representation.

\subsection{Spectral representation}

When defining the color of an object, we must not only specify the wavelengths it reflects, but also the ratio between the incoming energy and the outgoing energy at these wavelengths. The dependence of reflectance on the wavelength is called a \emph{reflectance spectrum}, and is usually a smooth, continuous curve.

Although this definition might be sufficient for reflective surfaces, describing the color emitted by a light source requires the knowledge of the source's power rather than reflectance. For these purposes, \emph{spectral power distribution} (SPD) is used. Generally, SPD is a function describing the relationship between wavelength and any radiometric or photometric quantity (radiant energy, luminance, luminous flux, irradiance et cetera\ldots). In this thesis, we will use SPD to describe the emissive properties of light sources, and will therefore consider SPD to be a function of wavelength and power.

To compute the color of an object under a light source, one must simply combine the light source's SPD with the reflectance curve of the object, as shown in figure. This way the physical properties of color are preserved and the result is the same as it would be in nature, which, as we will show later, is not always the case with tristimulus representation.

\subsection{Tristimulus representation} \label{ssec:tristimulusRepres}

The obvious drawback of spectral representation is the difficulty of its discretization. Another, bigger problem, is caused by the fact that there is an infinite number of possible spectral curves, but only a discrete number of colors perceptible by human eye or even possible to generate by a computer (use word domain?). Representing colors with spectral distribution therefore requires their conversion to a discrete space before arbitrary visualization process.

Tristimulus representation skips the conversion steps and saves the already discretized color as a set of three values. Although the original idea was to simulate the trichromatic perception of human eye (i.e. save values that specify how much have the red, green and blue cones been stimulated), over time, multiple other tristimulus color spaces have been created. They differ mostly in the range of colors they are capable of representing and in their practical use. Following, we provide an overview of some of the most popular ones.

\subsubsection{RGB color space}
The RGB color space is an additive space employing three primaries --- red, green and blue. In other words, if you have three lights with red, green and blue chromacities respectively and you use them to illuminate a single point, you can create any color within the RGB color space solely by changing the lights' intensities. 

An RGB value can be therefore thought of as a point in a 3-dimensional euclidean space with each of the coordinate axes representing one of the primaries. Specifically, as the light's intensities must be bounded, we can narrow this space down to a cube starting at the base of the coordinate system. Usually, the range for each value is defined within 0 and 255, but a normalized (0,1) range is also used.

Various implementations of the RGB color space exist. They differ in the specifications of the RGB primaries, and therefore in their \emph{color gamut}, which is the subset of colors they are capable of representing. Some examples (named in ascending order with respect to their color gamut) include ISO RGB, sRGB, Adobe RGB, Adobe Wide Gamut RGB and ProPhoto RGB. An illustrative comparison of the sRGB and Adobe RGB gamut in the chromaticity diagram (described thoroughly in~\cref{sssect:xyYcolorSpace}) can be seen in~\cref{fig:chromaticityDiagram}.

RGB color spaces are commonly used in everyday world, e.g. in LCD and LED displays, digital cameras, scanners and even in computer graphics rendering. Their main downside has, however, been discovered when designing color matching functions~\cite{colorMatchingDerivation}.

A \emph{color matching function} is a function designed to simulate the response of a certain type of cone in the human eye. In 1931, CIE designed a set of three color matching functions that could be used for spectral to RGB conversion~\cite{colorMatchingDerivation}. Denoted $\overline{r}(\lambda)$, $\overline{g}(\lambda)$ and $\overline{b}(\lambda)$, they approximate the response of the L, M and S cones respectively. However, as seen in figure~\cref{fig:colorMatchingRGB}, the functions may also acquire negative values. This posed a problem at that time due to calculation errors. Therefore, to eliminate these negative portions of functions, CIE designed a new, imaginary color space --- the XYZ color space (tu ref?).

\begin{figure}[t]
	\centering
	\begin{subfigure}{0.46\textwidth}
		\includegraphics[width=\linewidth]{img/matching_functions_rgb.png}
		\caption{ $\overline{r}(\lambda)$, $\overline{g}(\lambda)$ and $\overline{b}(\lambda)$ functions plotted with data by~\citet{colorMatchingRGBData}}
		\label{fig:colorMatchingRGB}
	\end{subfigure}
	\quad
	\begin{subfigure}{0.46\textwidth}
		\includegraphics[width=\linewidth]{img/matching_functions_xyz.png}
		\caption{$\overline{x}(\lambda)$, $\overline{y}(\lambda)$ and $\overline{z}(\lambda)$ functions according to their spectral data from~\citet{colorMatchingXYZData}}
		\label{fig:colorMatchingXYZ}
	\end{subfigure}
	\caption{Color matching functions}
	\label{fig:colorMatchingFunctions}
\end{figure}

\subsubsection{XYZ color space}

The XYZ color space is a hypothetical color space capable of encompassing all colors perceptible by the human eye. Its color matching functions, $\overline{x}(\lambda)$, $\overline{y}(\lambda)$ and $\overline{z}(\lambda)$, were specifically designed for the purposes of SPD to tristimulus conversion, which is computed using the following equations:
\begin{equation} \label{spdToXYZ}
	\begin{aligned}
	X=\int P(\lambda)\overline{x}(\lambda)d\lambda,\\
	Y=\int P(\lambda)\overline{y}(\lambda)d\lambda,\\
	Z=\int P(\lambda)\overline{z}(\lambda)d\lambda,\\
	\end{aligned}
\end{equation}
where $X$, $Y$ and $Z$ are the resulting tristimulus values and $P(\lambda)$ is the spectral power distribution.

Although the X, Y and Z primaries were designed so that the Y primary closely matches luminance and X and Z primaries give color information, they are only imaginary, i.e. they do not correspond to any spectral distribution of wavelengths. This property renders the whole XYZ space imaginary, which means that it cannot be used for visualization purposes. Its main function is to therefore serve as a ``middle step'' when performing a conversion from SPD to an arbitrary tristimulus space, which eliminates the need for other color matching functions. The conversion from XYZ into a tristimulus space can then be performed by a simple space-specific $3x3$ matrix transformation.

\subsubsection{xyY color space} \label{sssect:xyYcolorSpace}

In addition the impossible visualization process, another downside of the XYZ color space is that its values are practically unbounded and do not have any real meaning (such as the RGB triplets have). Therefore, a more intuitive color space has been created, which considers the relative proportions of the X, Y and Z values rather than their unbounded versions --- the xyY color space~\cite{xyYOverview}. It is based on the assumption that color can be regarded as a quantity with two properties: \emph{luminance} and \emph{chromaticity}.

First, the following conversion from the $X$, $Y$ and $Z$ values to their bounded versions, also called \emph{chromaticity coordinates}, is performed~\cite{xyYEquations}:
\begin{equation} \label{XYZtoxyY}
\begin{aligned}
&x=\dfrac{X}{X+Y+Z}\\
&y=\dfrac{Y}{X+Y+Z}\\
&z=\dfrac{Z}{X+Y+Z}\\
\end{aligned}
\end{equation}
Due to normalization ($x+y+z=1$), $z=1-x-y$, which means that we can drop the term $z$ from the representation as it does not give any additional information about the current color. It also implies that we lost some information during the conversion --- we cannot reconstruct the original XYZ triplet using only two values $x$ and $y$ and therefore cannot obtain the initial color. At least one of the original values is needed for this purpose ---~\citet{CIE} decided to use the $Y$ component, as it already specifies the luminance.

Plotting the values of the $x$ and $y$ components creates a \emph{chromaticity diagram}, shown in~\cref{fig:chromaticityDiagram}. Each point of the curved boundary line (which is also called the \emph{spectral locus}) corresponds to a XYZ value that is the result of a monochromatic radiation (i.e. a single-wavelength stimulus). All other chromaticities visible to the standard observer lie within a region bounded by the spectral locus.

\begin{figure}[t!]
	\centering
	\includegraphics[width=0.6\linewidth,height=0.3\textheight]{img/chromaticity_diagram.jpeg}
	\caption{An illustrative comparison of the sRGB and Adobe RGB gamut in the chromaticity diagram based on images created by~\citet{chromaticityDiagramResource}}
	\label{fig:chromaticityDiagram}
\end{figure}

\subsubsection{L*a*b*}

Although the xyY color space is already much more intuitive in terms of human color perception, the differences between individual triplets of the system are not perceptually uniform. The Hunter's Lab color space ref addressed this issue and was designed so that the distance between its two triplets characterized roughly how different they are in chromaticity and luminance. It is based on the Opponent color theory~\cite{opponentColorTheory}, which suggests that the cones in the human eye are linked together in opposing pairs and that the visual system records the \emph{difference} between the stimulation of the pairs rather than the cones' individual responses.

As the Hunter's Lab color space does not achieve perfect uniform spacing of values, CIE \emph{L*a*b*} color space (CIELAB) has been proposed in an attempt to improve some of its shortcomings and is now more widely used. However, neither of the systems are completely accurate in terms of perceptual uniformity~\cite{hunterLabCIELabComparison}.

The three opponent channels used to specify color in the CIE L*a*b* color space are defined as follows~\cite{labColorScale}:
\begin{itemize}
	\item \emph{L*} --- indicates lightness, i.e. the difference between \emph{light} and \emph{dark}. Its values range from $0$ (yielding black color) to $100$ (indicating diffuse white color)~\cite{labColorScale}.
	\item \emph{a*} --- defines the difference between \emph{green} and \emph{red}. Positives values of this component indicate the object's color to be more green, while negative values indicate red.
	\item \emph{b*} --- defines the difference between \emph{yellow} and \emph{blue}. Positive values indicate the object to be more yellow, while negative values indicate blue.
\end{itemize}
Neither the range of the a* nor the b* component has any specific numerical limits~\cite{labColorScale}.

The L*a*b* color space is a \emph{reference system} --- an abstract, non-intuitive space encompassing all the human perceptible colors. Due to its perceptual uniformity, it is used for color balance corrections by modifying the a* and b* components, and for lightness adjustments by modifying the L* component. 

Another advantage and common use of the L*a*b* color space is for computing \emph{color differences}. In 1976, CIE introduced the concept of \emph{Delta E}, which is the measure of change in visual perception of two colors~\cite{deltaEOverview}. Denoted $\Delta E_{ab}^*$, it is computed as an Euclidean distance between the two sample points, i.e.:
\begin{equation} \label{deltaE}
\Delta E_{76}=\sqrt{(L_{2}^* - L_{1}^*)^2 + (a_{2}^* - a_{1}^*)^2 + (b_{2}^* - b_{1}^*)^2},
\end{equation}
where $(L_{1}^*,a_{1}^*,b_{1}^*)$  and $(L_{2}^*,a_{2}^*,b_{2}^*)$ are the L*a*b* coordinates of the sample points.

However, the sensitivity of the human eye to color differences is not uniform. It is, for example, more sensitive to small color differences in dark blue colors than it is in e.g. light pastel colors. The $\Delta E_{ab}^*$ error does not take this ununiformity into account and therefore shows exaggerates differences in light colors while compressing perceptual distances between darker colors. To improve upon these shortcomings, other measuring techniques for computing Delta E have been proposed over the years, such as Delta94 and Delta2000.

\emph{Delta94} is computed by modifying the original L*a*b* values of both colors to compensate for perceptual distortions in the color space and computing Euclidean distance from the new modified values. Although the results match the human color difference perception more closely, the Delta94 error metric still lacks some accuracy in the blue-violet region~\cite{deltaEOverview}.
	
\emph{Delta2000} attempts to remove these inaccuracies. Along with the corrections added to Delta94, Delta2000 adds overall five correctional factors to the original $\Delta E_{ab}^*$ --- compensation factors for lightness, hue and chroma, compensation for neutral colors and, lastly, a hue rotation term for the problematic blue-violet regions. 

From the listed Delta E equations, the Delta2000 error measurements are the most accurate in terms of human color difference perception~\cite{deltaEOverview} and, therefore, will also be used in the practical parts of this thesis. However, as the specifics of the Delta2000 equations are out of scope of this thesis, we refer the interested reader to the original article by~\citet{delta2000}.

\subsubsection{Other color spaces}

In addition to the already named tristimulus color spaces, there exist many more used for various purposes. Following, we briefly overview some of them:
\begin{itemize}
\item \emph{L*u*v*} --- Similarly to the CIELAB system, L*u*v* (or CIELUV) aims for perceptual uniformity. As a matter of fact, the $L*$ value is defined in the same manner as in the CIELAB system, while $u$ and $v$ values are evaluated by certain projections of the $x$ and $y$ coordinates of the chromaticity diagram. When comparing their Euclidean error measure, the most important distinction between the two spaces is that while  CIELAB generally improves CIELUV in terms of color difference~\cite{CIELABcomparisonCIELUV}, CIELUV does not have as many inaccuracies in the dark regions~\cite{CIELABDarkSide}. Therefore, it is often recommended to use the CIELUV color space for characterization of color displays and CIELAB color space for the characterization of colored surfaces and dyes.

\item \emph{HSL} and \emph{HSI} color spaces define color by its \emph{hue}, \emph{saturation} and \emph{lightness} (or \emph{intensity}). They are an alternative representation of the RGB color space and must therefore be defined purely with reference to an RGB space~\cite{HSLreview}. As their components correlate better with human perception of color than those of the RGB system, they are often used in image processing applications, e.g. for processes such as feature detection (edge detection~\cite{edgeDetectionHSL}, object recognition) or image segmentation (which can be performed solely with/by? the hue component)~\cite{HSLreview}.

\item \emph{CMYK} model is a subtractive color model commonly used in color printing. It is based on RGB's complementary colors --- \emph{cyan}, \emph{magenta} and \emph{yellow} respectively. This means that assigning zero values to all components renders white light, and increasing the value of a component specifies how much of the respective color is \emph{subtracted} from the white light. Although the theory states that maximizing CMY values should render perfect black, in reality, the printing inks are not 100\% CMY and their combinations cannot produce rich black. For this purpose, a fourth component, \emph{black} ($K$), is often added, giving rise to the CMYK model.

\end{itemize}

Other color spaces include Munsell color system, RAL, Natural Color System, Pantone Matching System, CIELCH\textsubscript{ab}, CIELCH\textsubscript{uv}, etc$\ldots$

\subsection{Color representation in rendering}

Accurate color representation is the core of rendering softwares. Although most of today's renderers support multiple color spaces, we can still divide them into two main categories according to the space used during evaluation of light transfer equations --- \emph{tristimulus} and \emph{spectral} renderers.

Tristimulus renderers are usually based on the RGB color space, although they often offer conversions to other tristimulus spaces. Due to the ease of use and simplicity of representation, RGB renderers are more common in commercial rendering software. They provide realistically looking images, often indistinguishable from a photograph, and are more robust, easy to implement and memory efficient.

\begin{figure}[t]
	\centering
	{\sffamily
		\begin{tabular}{cc}
			\includegraphics[width=0.45\linewidth]{img/mitsuba_rgb_mode.jpg}
			&
			\includegraphics[width=0.45\linewidth]{img/mitsuba_spectral_mode.jpg}\\
		\end{tabular}
	}
	\caption{Comparison of an RGB-based rendering and spectral-based rendering as presented in the documentation of Mitsuba2~\cite{Mitsuba2}. Left: Spectral reflectance data of all materials is first converted to RGB and the scene is then rendered in the RGB mode, producing an unnaturally saturated image. Right: Scene is rendered directly in the spectral mode, resulting in more realistic colors. }
	\label{fig:mitsubaRGBSpectralComparison}
\end{figure}

However, light in real world does not travel as a tristimulus value, but rather as a distribution of wavelengths. As RGB renderers do not possess full-spectral information of materials and light in the scene, they cannot properly simulate the physical properties of the color during e.g. reflections or refractions when ray tracing.

Spectral rendering, on the other hand, uses full-spectral information of all materials and light in the scene during the whole rendering process. Obviously, before visualization occurs, spectral information must be converted into tristimulus (usually RGB) values, but this does not pose a problem as, at the moment of conversion, all the physically-based simulations have already taken place. Therefore, the rendered scene appears more realistic. We demonstrate this difference in~\cref{fig:mitsubaRGBSpectralComparison}, on a scene already rendered by Mitsuba2~\cite{Mitsuba2}.

In addition to rendering reflections and refractions more convincingly, another reason for using spectral rendering is its capability of simulating physically based phenomena that arises due to the interaction of color with light. Following, we overview some of the most common ones:
\begin{itemize}
\item \emph{Metamerism} \label{item:metamerism}

As already mentioned in~\cref{ssec:tristimulusRepres}, the human tristimulus perception has a significantly lower domain than the (practically infinite) spectral domain. Therefore, two different spectra can trigger the same cone response in the human eye and appear to have the same color (and, subsequently, to have the same RGB values), giving rise to a phenomenon called \emph{metamerism}. The two spectra evaluating to the same tristimulus values are called \emph{metamers}.

In real world, metamerism is often perceived when the lighting conditions under which we observe metamers change. An example of this can be seen in~\cref{fig:metamerism}. Although we perceive the color of both objects to be the same under D65 illuminant (daylight), when illuminating the scene with x illuminant, the color changes.

Obviously, this behavior is irreproducible by an RGB renderer, as it cannot replicate the behavior of spectral reflectance under an illuminant.

\begin{figure}[t]
	\centering
	{\sffamily
		\begin{tabular}{cccc}
			Spectral curves & D65 Illuminant & FL11 Illuminant 
			\vspace{1em} \\
			\includegraphics[width=.21\linewidth]{img/noInter.png}
			&
			\includegraphics[width=.21\linewidth]{img/noInter.png}
			& 
			\includegraphics[width=.21\linewidth]{img/noInter.png}
			& 
			\includegraphics[width=.21\linewidth]{img/noInter.png}
			\vspace{1em} \\
			\includegraphics[width=.21\linewidth]{img/noInter.png}
			&
			\includegraphics[width=.21\linewidth]{img/noInter.png}
			&
			\includegraphics[width=.21\linewidth]{img/noInter.png}
		\end{tabular}
	}
	\caption{The effects of metamerism. Left: Two different spectral reflectance curves, both corresponding to RGB=(0,255,0). Middle: A box in a cornell box rendered with a 
		A Cornell box rendered with }
	\label{fig:metamerism}
\end{figure}


\item \emph{Fluorescence}

By definition, fluorescence occurs when light from one excitation wavelength $\lambda_0$ is absorbed by an object and is almost immediately re-emitted at a different, usually longer, wavelength $\lambda_1$~\cite{fluorescenceDefinition}. Specifically interesting is the fact that the absorbed light can come from outside of the visible spectrum and be re-emitted inside it, which results in unrealistically bright material appearance, perceivable in real world when for example fish, corals, jellyfish or even minerals are illuminated by a UV light.

RGB renderers attempt to fake this kind of behavior it through custom shaders~\cite{fluorescencePolarization}. As it produces satisfactory results and is immensely easier to implement than physical simulation, physically based fluorescence has received small amount of work. Its support can be found in spectral renderers, added for example to ART by~\citet{fluorescenceART}.

\item \emph{Iridescence}

\emph{Iridescence}, or goniochromism, is a phenomenon occurring when certain surfaces change their color according to the current viewing angle. It arises when the object's physical structure causes interferences between light waves (e.g. inside extremely thin dielectric layers), yielding rich color variations~\cite{iridescenceArticle1}. It can be perceived in nature in certain plants, specific minerals, butterfly wings, peacock's feathers, snakes, but also in man-made products such as oil leaks, soap bubbles or car paints.

Similarly to fluorescence, iridescent behavior can be ``faked'' in an RGB renderer~\cite{iridescenceRGB}. However, research based on physical properties of iridescence has also been conducted. For further information about the current development, we refer the interested reader to the articles by~\citet{iridescenceArticle1},~\citet{iridescenceArticle2}, or~\citet{iridescenceArticle3}.

\item \emph{Dispersion}

When light travels from one medium to another (e.g. when light hits glass or water), its direction of travel is changed. This phenomenon is called \emph{refraction} and is closely described by Snell's law, which specifies how the angle of refraction can be computed from the angle of incidence and the \emph{refraction indices} of the two media~\cite{snellsLaw}. However, the refraction index depends not only on the \emph{type} of media, but also on the current \emph{wavelength}~\cite{dispersionRendering1} --- which implies that the resulting direction of photons of different wavelengths might vary.

Probably the most popular example of this phenomena is white light hitting a dispersive prism. Upon interaction, light is split into a spectrum, creating a ``rainbow'' effect.

There have been multiple attempts to simulate physically-based dispersion. We refer the interested reader to articles by~\citet{dispersionRendering1} or~\citet{dispersionRendering2}.

\item \emph{Polarization}

Electromagnetic waves traveling through space are \emph{transverse waves} --- their oscillation is perpendicular to their path of propagation. By default, the directions of oscillations are arbitrary for each photon --- this type of light is called an \emph{unpolarized light}. Restrictions to the directions of oscillations (also called \emph{polarization}) render \emph{polarized light}. Such phenomenon usually occurs upon light's interaction with certain materials.

The polarization process contributes to the overall color only in special cases (e.g. when using polarization filters)~\cite{fluorescencePolarization}. Therefore, it receives little attention in implementation of rendering softwares. However, for physical consistencies (and due to the possibility of special scenes) both ART~\cite{ART} and Mitsuba~\cite{Mitsuba2} follow the direction of oscillation during the rendering process.
\end{itemize}

Other researched phenomena (some of it closely linked to the already mentioned ones) include \emph{phosphorescence}, \emph{bioluminescence}, \emph{dichroism}, \emph{opalescence}, \emph{aventurescence} and many more.


\chapter{Spectral Uplifting}



Maybe call the next section spectral sampling? 


Spectrum to RGB conversion, mention the problems, inaccuracies - depending on the length of text, the intro to spectral uplifting (upsampling) can be regarded as a separate section

Also, probably here is a good time to mention that we are focusing on the reflectance spectra, however we will also talk about emission spectra in this section solely for research purposes

\section{Spectral color representation}


\subsection{Available methods}

Maybe separate them into subsections, or add a subsection for comparing the results

\subsection{Trigonometric moment method}

A review of the moment method (basically just a review of the paper)

\subsubsection{Evaluation of various parameters}

Add the results from the tests I ran back in April - which combination of number of moments, Warp/NonWarp and Mirror/NonMirror techniques is the most optimal. Also mention that we do not want to use complex moments as it doubles the space needed which will be unusable for the optimizer in the Implementation section. 

\subsubsection{Reconstruction results}
Try the reconstruction on various spectrum values, also add the respective RGB values.

It might be nice to also add the ideal coefficients that we got from the borgtool by optimizing and comparing them to the coefficient that were originally computed (these are different mainly due to rounding errors during the algorithm). 

\section{Uplifting Methods}

alebo rozdelit a urobit chapter samostatny ze spectral uplifting a samostatny kde bude spectral color representation
\chapter{Implementation} \label{chap:implementation}

We approach the problem of spectral uplifting similarly to~\citet{upsamplingJakobHanika}, where an uplifting model is created prior to rendering. Our implementation therefore consists of two parts --- \emph{model creation} and its subsequent \emph{utilization} in a rendering software. 

For the first part, we extend an already existing uplifting tool, Borgtool, which is currently used for creating sigmoid-based RGB cubes in a way as described in~\cref{alg:upliftingAlgSigmoid}. We add the possibility for creating trigonometric moment-based cubes (from now on referred to as \emph{trigonometric moment cube}), i.e. for the spectra to be stored with trigonometric moments rather than sigmoid coefficients. We also add an option for constraining such a cube with a user-specified constraint set (e.g. a color atlas).

We then describe the theory behind the integration of this model into a rendering software. In practice, we add its support into ART, which, up until now (version 2.0.3), has used only one built-in sigmoid-based cube for uplifting purposes.

\section{Uplifting model}

As we base most of our implementation on the already existing sigmoid-based approach, we start this section with the detailed description of its model, i.e. the sigmoid-based cube. We then describe our trigonometric moment cube, which can be viewed as its extension.

The sigmoid-based cube structure contains multiple entries in form of evenly spaced lattice points. Following, we name the main parameters of a single cube entry:
\begin{itemize}
	\item \emph{target RGB} --- the RGB coordinates that the point has in the cube
	\item \emph{coefficients} --- 3 sigmoid coefficients used to reconstruct a spectrum so it matches the target RGB
	\item \emph{lattice RGB} --- the actual RGB that the reconstructed spectrum evaluates to. Ideally, this should match the target RGB
\end{itemize}
Along with its entries, the resulting cube structure also stores a few other properties, both \emph{static}, such as the illuminant according to which the RGB cube is uplifted, and \emph{user-adjustable}, such as the cube dimension or the fitting threshold (i.e. the maximum allowed difference between the target and the lattice RGB).

Our trigonometric moment cube includes most of these parameters, and mainly extends the ones that are not suitable for the moment representation. The main difference between the two cubes lies in the distinction of the lattice points --- while the sigmoid cube regards all of its points as equal, the trigonometric moment cube distinguishes (by means of a \texttt{seeded} boolean parameter per entry) between \emph{seeded points}, i.e. the lattice points that store the user-inputted RGB:spectra mappings; and \emph{regular points}, which do not.

Our requirements for the shape of the spectra at seeded points differ from the ones at regular points. While we prefer the regular points to have their spectra as smooth as possible in order to avoid unexpected artifacts under other illuminants, the coefficients of the seeded points must reconstruct spectra almost identical to the input spectra, which might include sharp edges and spikes.

Therefore, it is sufficient for the regular points to be represented with a smaller number of coefficients, while seeded points might require a lot more. Although a smooth spectrum can be represented with a high number of coefficients, such a representation is memory inefficient, its reconstruction is more time consuming, and, most importantly, it does not work well with the optimizer. Based on our experiments in~\cref{ssec:noOfMoments}, we decide to store the spectra of regular points with 3 coefficients and adjust the number of coefficients of the seeded points depending on the nature of its desired spectral shape.

In addition to supporting variable number of coefficients (ranging from 3 to 21), the trigonometric moment cube also supports the possibility of having multiple coefficient representations per lattice point. The sole purpose of this extension is to lower the cube size requirements upon constraining, which we discuss later in~\cref{ssec:cubeSeeding}.

In addition to the cube structure, the construction of the trigonometric moment cube is also similar to the one of the sigmoid cube, which follows~\cref{alg:upliftingAlgSigmoid}. Its main distinctions are in the constraining process (which the sigmoid cube lacks) and in the first round of optimizing, or, as we refer to from now on, \emph{fitting}.

Following, we name the individual steps of the process, which we then describe in greater detail.

\begin{enumerate}
	\item \emph{Initialization}
	\item \emph{Cube seeding (optional)}
	\item \emph{Fitting of starting points}
	\item \emph{Cube fitting}
	\item \emph{Cube improvement}
	\item \emph{Cube storage}
\end{enumerate}

\subsection{Initialization} \label{ssec:initialization}

This part of the run is responsible for the following:
\begin{itemize}
	\item parsing of the parameters
	\item initialization of the cube and its entries with default values
	\item loading of the required constraint sets
\end{itemize}

The initialization of the cube is pretty straightforward, as all of its properties are either user-defined or set to default (note: the default illuminant is always D65). The number of cube entries is directly proportional to the cube's \texttt{dimension} parameter, which specifies the number of entries per one axis. This renders the total number of entries to $dimension^3$. As the lattice points are positioned evenly, their target RGB values are equivalent to their coordinates in the RGB cube.

The constraint sets are inputted in a form of a simple .txt file, which contains merely a list of entries in a textual form as shown in~\cref{fig:macbethSampleText}. This step is responsible for the parsing of the file, and for storage of individual constraints. In order to avoid high memory requirements arising with large input datasets, we do not store the spectral data directly, but take advantage of the trigonometric moments.

\begin{figure}
	\lstset{
		string=[s]{"}{"},
		comment=[l]{:},
		commentstyle=\color{black},
		basicstyle=\scriptsize
	}
	\begin{lstlisting}[label=lst:atlasEntry]
	Entry ID:   orange
	------------------------------------------------------------------------
	Description           :  "orange" patch of the Macbeth colour checker
	Type                  :  reflectance spectrum
	Fluorescence data     :  no
	Measurement device    :  
	Measured by           :  
	Measurement date      :  
	
	Sampling information
	--------------------
	Type	    	      :  regular
	Start                 :  380.0 nm
	Increment             :  5.0 nm
	Maximum sample value  :  100.0
	
	ASCII sample data
	-----------------
	{6.143748,  5.192119,  4.867970,  5.092529,  4.717562,  4.663087, 
	4.455331,  4.562958,  4.517197,  4.536289,
	4.454180,  4.543101,  4.491708, ... }
	\end{lstlisting}
	\caption{A sample entry from the Macbeth Color Checker atlas}
	\label{fig:macbethSampleText}
\end{figure}

We store the spectral curves of the individual constraints with Fourier coefficients as described in~\cref{par:spectrumToCoefficientConversion}. We mirror but do not warp the signal prior to coefficient computation (see~\cref{sec:storingMoments}).

The number of coefficients per constraint variable, ranging from 4 to 21. We explain our method for determining the sufficiency of coefficient representation, and therefore the coefficient count for each constraint, in~\cref{ssec:noOfMoments}.

Note that we require the constraints to have identical sampling information. This representation is used internally throughout both the fitting and the uplifting process for e.g. spectral reconstruction for the purposes of color conversions, or for coefficient recalculation mentioned later in~\cref{ssec:cubeFitting}.

\subsection{Cube seeding} \label{ssec:cubeSeeding}

In order to uplift the whole cube as described in~\cref{alg:upliftingAlgSigmoid}, we must first fit one or more \emph{starting points} whose coefficients can then be used as prior for the fitting of other lattice points. For this purpose, we utilize the user-specified constraint set. The general idea behind this process is to copy the coefficients of constraints to specific lattice points, and then use these coefficients as prior for fitting said lattice points. We refer to this process as \emph{seeding} of the cube, and term the constraints assigned to these lattice points their \emph{seeds}.

The ideal scenario would be if the RGB values of the constraints were to perfectly match the coordinates of the lattice points. However, as the constraints can evaluate to virtually any triplet within the $[0,1]$ range, it is most likely that they would correspond to points inside the cube's voxels.

A realistic approach would be to create a complete injective mapping between the constraints and their closest lattice points. However, such a mapping would provide satisfactory results only if we were to use the nearest-neighbor method for uplifting non-mapped RGB triplets. As we disregard the nearest-neighbor approach due to its inaccurate results (see~\cref{fig:sigmoidTexture}), we must use either the interpolation of coefficients or spectra, which both employ data at all 8 voxel corners during the uplift. Seeding only one of the 8 points might therefore cause the spectral curves of the other 7 to be considerable distinct from the original constraint. This is mainly due to our choice of coefficient count for non-mapped entries, which is a lot lower than for the seeded entries (see thorough explanation in~\cref{ssec:noOfMoments}), i.e. the curves of the non-seeded entries cannot be as precise. 

We can observe this behavior in~\cref{fig:seedingMethod1corner}, where we provide the curves reconstructed at the 8 corners of a voxel and compare their trilinear interpolation to the original constraint. The original constraint significantly differs from the uplift, which may cause color artifacts under different illuminating conditions. However, as seen in~\cref{fig:seedingMethod8corners}, propagating the information about the original reflectance of the constraint to all voxel corners improves the result remarkably. We therefore opt for seeding all 8 voxel corners per constraint.

\begin{figure}[t]
	\centering
	\begin{subfigure}[t]{0.54\textwidth}
		\includegraphics[width=\linewidth]{img/seeding_method_legend.png}
	\end{subfigure} \\
	\begin{subfigure}[t]{0.45\textwidth}
		\includegraphics[width=\linewidth,height=0.2\textheight]{img/seeding_method_1corner.png}
		\caption{Only 1 voxel corner seeded}
		\label{fig:seedingMethod1corner}
	\end{subfigure} \hspace{0.1em}
	\begin{subfigure}[t]{0.45\textwidth}
		\includegraphics[width=\linewidth]{img/seeding_method_8corners.png}
		\caption{All 8 voxel corners seeded}
		\label{fig:seedingMethod8corners}
	\end{subfigure}
	\caption{Uplifting results according to the seeding method}
	\label{fig:seedingMethodInterpolation}
\end{figure}

During the seeding process, it may occur that two constraints would fall into neighboring voxels, i.e. that they would share some of the voxel corners. In order to utilize both of these constraints as seeds, we support the possibility of one lattice point having multiple coefficient representations. In addition to coefficients and their count, we also store an entry ID per each representation, so as to later distinguish the reconstructed curves during rendering and decide which to employ (explained in more detail in~\cref{sec:rendererIntegration}).

If two constraints fall into the same voxel, however, there is no way of determining the interpolation of which the user desires upon uplifting the RGB values inside said voxel. We therefore discard one of the constraints and throw an error informing the user of the collision and suggesting the increase of the cube \texttt{dimension} parameter.

We show an example of a cube seeded and subsequently fitted with the Munsell Book of Color used as a constraints set in~\cref{fig:seededCubeMCB}. Lattice points marked as black represent the seeded points. Note that a lot of these points store multiple coefficient representations.

\begin{figure}[t!]
	\centering
	\captionsetup[subfigure]{font=footnotesize,labelfont=footnotesize}
	\captionsetup[subfigure]{justification=centering}
	\begin{subfigure}[t]{0.22\textwidth}
		\includegraphics[width=\linewidth]{img/seededCube_mcb1.png}
		\label{fig:seededCube_mcb1}
	\end{subfigure} \hspace{0.05em}
	\begin{subfigure}[t]{0.22\textwidth}
		\includegraphics[width=\linewidth]{img/seededCube_mcb2.png}
		\label{fig:seededCube_mcb2}
	\end{subfigure} \hspace{0.05em}
	\begin{subfigure}[t]{0.22\textwidth}
		\includegraphics[width=\linewidth]{img/seededCube_mcb3.png}
		\label{fig:seededCube_mcb3}
	\end{subfigure} \hspace{0.05em}
	\begin{subfigure}[t]{0.22\textwidth}
		\includegraphics[width=\linewidth]{img/seededCube_mcb4.png}
		\label{fig:seededCube_mcb4}
	\end{subfigure}
	\caption{A 32-dimensional cube fitted with the Munsell Book of Color. Note that, due to the low dimension of the cube, multiple collisions occurred, i.e. not all constraints have been utilized.}
	\label{fig:seededCubeMCB}
\end{figure}

Constraining the uplifting process is optional. If no constraint set is inputted, all lattice points are regarded as regular points and the cube is fitted from the middle in the same manner as the sigmoid cube. The resulting uplifting structure provides no advantages over the sigmoid cube, other than having slightly different spectral shapes.

Supporting this option requires us to specify 3 prior coefficients for the center point (i.e. the point corresponding to an RGB of~$(0.5, 0.5, 0.5)$). By storing spectral curves that roughly evaluate to such RGB with the trigonometric moments, we observe that the values of the coefficients are approximately $\{0.5, 0, 0\}$. Therefore, we use them as prior.

We provide a comparison between the starting points' placement when seeding from the middle (either with our or the sigmoid method) and when seeding with a constraint set in form of the Munsell Book of Color in~\cref{fig:seededStartingPoints}.

\begin{figure}[t]
	\centering
	\captionsetup[subfigure]{font=footnotesize,labelfont=footnotesize}
	\captionsetup[subfigure]{justification=centering}
	\begin{subfigure}[t]{0.45\textwidth}
		\includegraphics[width=\linewidth]{img/seededStarting_sigmoid.png}
		\label{fig:seededStarting_sigmoid}
	\end{subfigure} \hspace{0.2em}
	\begin{subfigure}[t]{0.45\textwidth}
		\includegraphics[width=\linewidth]{img/seededStarting_mcb.png}
		\label{fig:seededStarting_mcb}
	\end{subfigure}
	\caption{Comparison of the position of starting points when fitting from the middle with sigmoids (left)
	 and when seeding with the Munsell Book of Color (right)}
	\label{fig:seededStartingPoints}
\end{figure}

\subsection{Fitting of starting points} \label{ssec:startingPointsFitting}

By seeding the cube, we have appointed coefficients to some of the lattice points. These coefficients reconstruct a spectrum that evaluates to an RGB value which we denote as the \emph{lattice RGB}. The difference between the lattice and the target RGB is therefore equal to the distance between the lattice point and its assigned constraint in the cube. It is apparent that the distance may be higher than the defined fitting threshold. In such cases, we must ``improve'' upon the coefficients so that the resulting color difference is as low as possible.

Our problem of improving the coefficients satisfies the definition of the \emph{Non-linear Least Squares} problem~\cite{nonLinearLeastSquares}. Non-linear Least Squares is an unconstrained minimization problem in the following form:
\begin{equation} \label{eq:nonLinearLeastSquares}
\underset{x}{\text{minimize}} \hspace{0.5em} f(x) = \sum_{i} f_{i}(x)^{2},
\end{equation}
where $x= \{x_{0}, x_{1}, x_{2}, ... \}$ is a parameter block that we are trying to improve (i.e. our coefficients) and $f_{i}$ are so-called \emph{cost functions}. The definition of cost functions is dependent solely on the current problem. In our case, we primarily require to minimize the difference between the lattice and the target RGB. Our secondary requirement is for the shapes of the curves of the original constraint and the of the seeded point to be similar. This gives rise to multiple choices for cost functions, such as using the difference between curves along with the Delta E error, using one or multiple cost functions for the RGB error etc\ldots After implementing some of them and testing their performance, the results of which we provide in~\cref{ssec:costFunctions}, we decided on using four cost functions --- three for specifying the absolute difference in one of the three axes of the cube, and one for specifying the average distance between curves per wavelength sample. We also include a heuristic which sets the value of the fourth cost function to 0 if the curve distance falls below a certain threshold, and iteratively increases the threshold if the optimizer fails. The reasoning behind this is also explained in~\cref{ssec:costFunctions}.

To solve an optimization problem defined in the way as described above, we use, similarly to~\citet{upsamplingJakobHanika} and~\citet{upsamplingFluorescence}, CERES solver.

\subsubsection{CERES solver} \label{sssec:ceresSolver}

As already mentioned in~\cref{sec:upliftingMethods}, CERES solver is an open-source library for solving large optimization problems such as the Non-linear Least Squares problem. It consists of two parts --- a \emph{modeling API}, which provides tools for the construction of optimization problems, allowing us to set parameters such as maximum number of iterations of the optimizer or maximum number of consecutive nonmononotic steps; and a \emph{solver API} that controls the minimization algorithm.

To solve a Non-linear Least Squares problem, the solver requires us to specify only a so-called \emph{residual block}, which is a structure defined by the prior coefficients and the cost functions. During the execution, the solver attempts to minimize the values of the cost functions (or \emph{residuals}) in the residual block. The execution is aborted and the current best parameter block returned when the solver achieves either the specified number of iterations or nonmonotonic steps. For more information on the specifics of CERES solver, we refer the interested reader to its documentation by~\citet{ceresNonLinearLeastSquares}.

There is one main downside to using CERES solver. As it was designed to handle very large, sparse problems where every residual term depends on only a few of the input parameters, it is not ideal for solving problems with only one large parameter block, i.e. it might get stuck in local minima and therefore produce unsatisfactory results. Unfortunately, if a seeded point is represented with a high number of coefficients, our optimization problem falls into this category.

We solve such problematic cases by applying a simple heuristic, which consists of slightly altering the first coefficient (as it has the highest influence on the shape of the curve) and running the optimizer again. However, although such an optimization greatly improves the overall performance of fitting, it remains insufficient for too high a number of coefficients, i.e. the threshold for the fourth residual must be increased to values extremely high and, by then, it loses resemblance to the original shape.

Therefore, we implement another heuristic improvement --- if the coefficient count is higher than 14, we let the optimizer optimize only the first 4 coefficients while leaving the others constant. We use the threshold of $c > 14$ as that is roughly the boundary where the fitted curves begin to show undesired artifacts, and we optimize the first 4 coefficients because their number is both low enough for the optimizer to handle without errors, and high enough so we give the fitting process enough degrees of freedom.

We summarize the fitting process of the starting points, including the utilization of threshold for our fourth cost function, in~\cref{alg:fittingAtlasLatticePoints}. If the optimizer is unable to fit a seeded points, we throw a warning and convert it into a regular point. However, as of yet, we have not encountered a failure.

\begin{algorithm}[t!]
	\caption{Fitting of one coefficient representation of a $point$ from seeded points}
	\label{alg:fittingAtlasLatticePoints}
	\begin{algorithmic}[1]
		\State $threshold \gets 0.001$
		\While {$threshold < 1$}
		\State $coefsToFit \gets$ either the first 4 coefficients or all of them depending on the coefficient count of $point$
		\State $i \gets 0$
		\While {optimizer is unsuccessful \textbf{and} $i < maxIterations$}
		\State heuristically change the first coefficient of $coefsToFit$
		\State run the optimizer with parameters $coefsToFit$ and threshold set to $threshold$
		\State $i++$
		\EndWhile
		\If{optimizer was successful}
		\State $point.coefs \gets coefsToFit$
		\State break
		\EndIf
		\State increase $threshold$
		\EndWhile
	\end{algorithmic}
\end{algorithm}

\subsection{Cube fitting} \label{ssec:cubeFitting}

As the seeded points are represented with a higher number of coefficients and may even contain multiple coefficient representations, they cannot be directly used as prior for the regular points. First, they must be ``converted'' into a lower-dimensional representation.

We refer to the conversion process as \emph{coefficient recalculation}. It consists of reconstructing the reflectance spectrum of the seeded point and subsequently saving it with 3 coefficients. Although this process causes significant loss of spectral information, it preserves the rough outline of the curve. This works to our benefit --- it reduces the likelihood of significant color artifacts between the seeded points and regular points while keeping the spectra smooth.

\begin{figure}[t!]
	\centering
	\captionsetup[subfigure]{font=footnotesize,labelfont=footnotesize}
	\includegraphics[width=0.8\linewidth]{img/recalculation.png}
	\caption{An illustrative demonstration of a problem posed by the coefficient recalculation of a seeded point containing multiple metameric spectra (stored in the form of moment representations)}
	\label{fig:recalculation_process}
\end{figure}

A problem arises if the seeded point that is being recalculated for the purposes of fitting a regular point contains multiple moment representations from different metameric families. We illustrate this situation in~\cref{fig:recalculation_process}. The recalculated seeded point (denoted $x$) contains the moment representation of both the constraint $A$ and $B$. As these have vastly distinct shapes, it is natural that they evaluate to completely different colors under error-prone illuminants (specifically, the colors shown in the image are under FL11). Choosing to recalculate only the representation of $A$ in order to obtain the prior coefficients of the regular point (denoted $y$) would result in significant color artifacts between $y$ and the voxel seeded with $B$. Symmetrically, the same applies to choosing to recalculate only the coefficient representation of $B$. We can also observe this in~\cref{fig:recalculation_colorGradients}, where we present the comparison of the color gradients created by respectively using the individual recalculation techniques. Note that the spectra (and, subsequently, colors) that the $A$ and $B$ entries evaluate to in both~\cref{fig:recalculation_process} and~\cref{fig:recalculation_colorGradients} are not the original spectra of the constraints, but the spectra that the coefficient representations saved at the seeded point $x$ evaluate to. Also note that the spectra and colors corresponding to the point $y$ are the recalculation results, not the final fitting results.

\begin{figure}[t!]
	\centering
	\captionsetup[subfigure]{font=footnotesize,labelfont=footnotesize}
	\captionsetup[subfigure]{justification=centering}
	\begin{subfigure}[t]{0.30\textwidth}
		\includegraphics[width=1\linewidth,height=4em]{img/recalculation_color_green.png}
		\includegraphics[width=1\linewidth,height=4em]{img/recalculation_color_fitGreen.png}
		\includegraphics[width=1\linewidth,height=4em]{img/recalculation_color_red.png}
		\caption{Recalculation of the coefficient representation of $A$ only}
		\label{fig:recalculation_colorGradients_green}
	\end{subfigure}
	\begin{subfigure}[t]{0.30\textwidth}
		\includegraphics[width=1\linewidth,height=4em]{img/recalculation_color_green.png}
		\includegraphics[width=1\linewidth,height=4em]{img/recalculation_color_interpolated.png}
		\includegraphics[width=1\linewidth,height=4em]{img/recalculation_color_red.png}
		\caption{Recalculation of the interpolation between\\ $A$ and $B$}
		\label{fig:recalculation_colorGradients_interpolated}
	\end{subfigure}
	\begin{subfigure}[t]{0.30\textwidth}
		\includegraphics[width=1\linewidth,height=4em]{img/recalculation_color_green.png}
		\includegraphics[width=1\linewidth,height=4em]{img/recalculation_color_fitRed.png}
		\includegraphics[width=1\linewidth,height=4em]{img/recalculation_color_red.png}
		\caption{Recalculation of the coefficient representation of $B$ only}
		\label{fig:recalculation_colorGradients_red}
	\end{subfigure}
	\caption{Color gradients resulting from our experiment in~\cref{fig:recalculation_process}. For each figure, the upper and the lower patch stand for the colors of coefficient representations of $A$ and $B$ respectively, while the middle patch stands for the color achieved from the current recalculation technique.}
	\label{fig:recalculation_colorGradients}
\end{figure}

As it is our intention to keep the color transitions within all voxel pairs smooth, we propose the interpolation of spectra reconstructed from the moment representations.

\subsubsection{Interpolation of metamers}
\label{sect:ims}
In the following, we show that the linear combination of two spectra that are metameric under a given light source results in another metameric spectrum. To our best knowledge, this insight, while not particularly mathematically complex, has not been explicitly stated in graphics literature before.

Let us assume the spectral power distributions of two metamers saved at a lattice point, $P_1(\lambda)$ and $P_2(\lambda)$, that satisfy the conditions
\begin{equation} 
\begin{aligned}
\int P_1(\lambda)\overline{r}(\lambda)d\lambda=\int P_2(\lambda)\overline{r}(\lambda)d\lambda\\
\int P_1(\lambda)\overline{g}(\lambda)d\lambda=\int P_2(\lambda)\overline{g}(\lambda)d\lambda\\
\int P_1(\lambda)\overline{b}(\lambda)d\lambda=\int P_2(\lambda)\overline{b}(\lambda)d\lambda\\
\end{aligned}
\label{equation:metamers}
\end{equation}
where $\overline{r}(\lambda)$, $\overline{g}(\lambda)$ and $\overline{b}(\lambda)$ are the RGB color matching functions.

Let us express the R component of the RGB value resulting from the linear combination of $P_1(\lambda)$ and $P_2(\lambda)$ as follows:
\begin{equation}
\begin{aligned}
R = \int a \cdot P_1(\lambda)\overline{r}(\lambda)d\lambda + 
b \cdot P_2(\lambda)\overline{r}(\lambda)d\lambda,\\
\text{where}\ a + b = 1
\end{aligned}
\end{equation}

By rewriting this expression and utilizing the equality from~\cref{equation:metamers}, we get
\begin{equation*}
\begin{aligned}
R = a \cdot \int P_1(\lambda)\overline{r}(\lambda)d\lambda +
(1-a) \cdot \int P_1(\lambda)\overline{r}(\lambda)d\lambda,\\
\end{aligned}
\end{equation*}
So
\begin{equation*}
\begin{aligned}
R = \int P_1(\lambda)\overline{r}(\lambda)d\lambda\\
\end{aligned}
\end{equation*}
The same proof can be equivalently applied to the G and B components of the resulting RGB value. Therefore, we conclude that the resulting spectral distribution is also a metamer.

We use this observation in order to achieve smoother color transitions between distinct metameric families by interpolating between metameric spectra stored (in the form of moment representations) at lattice points that contain multiple coefficient representations.

Additionally, we use to obtain the lattice RGB of such points --- i.e. we store the RGB of the interpolated spectrum. Although this information is meaningless for the purposes of further fitting, it gives us an approximation of how well the individual points are fitted.

Other than the coefficient recalculation, our fitting process is carried out in a manner similar to that of the sigmoid fitting (see~\cref{alg:upliftingAlgSigmoid}), where the lattice points are fitted in multiple \emph{fitting rounds}, each round attempting to fit the neighbors of the already fitted points. We provide a more detailed description of the principle behind the fitting algorithm used in our implementation in~\cref{alg:upliftingAlgMoments}. Note that each round of the fitting is multithreaded, which substantially increases time performance.

\begin{algorithm}[t!]
	\caption{Fitting of the cube from starting points}
	\label{alg:upliftingAlgMoments}
	\begin{algorithmic}[1]
		\State $fittingRound \gets$ $0$
		\State $unfittedPoints \gets$ a list of all points in $RGBCube \setminus startingPoints$
		\ForAll{$point \in unfittedPoints$}
		\State $point.fittingDistance = MAX\_DOUBLE$
		\EndFor
		\While {$unfittedPoints$ is not empty}
		\State{$currRoundPts \gets$ points from $unfittedPoints$ that have at least one fitted neighbor}
		\ForAll{$point \in currRoundPts$}
		\ForAll{$fittedNeigbor \in $ fitted neighbors of $point$}
		\If{$fittedNeighbor \in seededPoint$}
		\State{$point.coefs \gets$ recalculateCoefs($fittedNeighbor.coefs$)} \label{algStep:coefficientRecomputation}
		\Else
		\State $point.coefs \gets fittedNeighbor.coefs$
		\EndIf 
		\State $[sDist,sCoefs]\gets$ CERES.Solve($point.coefs$, $costFunctions$)
		\If{$sDist \leq point.fittingDistance$}
		\State $point.fittingDistance \gets sDist$
		\State $point.coefs \gets sCoefs$
		\EndIf
		\If{$cDist \leq fittingThreshold$}
		\State $point.treated = true$
		\State break
		\EndIf
		\EndFor
		\If{$point.fittingDistance > fittingThreshold$ \textbf{or} $point$ has tried the coefficients of all of its neighbors}
		\State remove $point$ from $unfittedPoints$
		\EndIf
		\EndFor	
		\State $fittingRound \gets fittingRound+1$
		\EndWhile
	\end{algorithmic}
\end{algorithm}

\subsection{Cube improvement} \label{ssec:cubeImprovement}

In extreme cases, such as when using a low fitting threshold or a sparsely-sampled constraint set, the fitting of some points may be unsuccessful. We assign most of these failures to the shortcomings of ART, since its functions are employed in the Borgtool for color conversion purposes.

Specifically, the conversion of an equal energy reflectance spectrum to RGB under the D65 illuminant does not produce the expected $RGB = (255, 255, 255)$, but rather an RGB of $(254.95, 255.005, 255.0003)$ for a 1nm sample increment, and, even worse $(254.88, 255.07, 254.93)$ for a 10nm increment. To force ART to reproduce an RGB of $(255, 255, 255)$, a spectrum with slightly lower values in the area around 550nm is required. Therefore, although a coefficient set $c = {1, 0, 0}$ represents an equal energy spectrum, it does not suffice for the optimizer. Additionally, 3 trigonometric coefficients are not capable of representing the slightly modified spectrum which ART regards as equal energy spectrum without slight error. This becomes even more visible if the amount of samples used for the internal representation of spectra is low, as it gives less freedom to the optimizer. Furthermore, this problem may also arise for target RGB values extremely close to $(255, 255, 255)$ (and not only the lattice point with RGB=$(255,255,255)$), which is mainly the case in higher-resolution cubes.

To minimize the created errors, we add a heuristic-based improvement of the coefficients, which sets their values to ones we assume are closest to the optimum and then proceeds similarly to the coefficient improvement of seeded entries. For a pre-defined amount of times, it slightly changes up the coefficients and runs the optimizer again, terminating if successful. However, neither this, nor any other heuristic-based improvement approach we attempted to implement, were capable of completely eliminating failures. Their only asset was a slightly lowered fitting threshold in some of the cases.

However, we note that these shortcomings are extremely rare and do not visibly lower the accuracy of the uplifting model.

\subsection{Cube storage}

Once the cube is fitted, its contents are written to a binary file. As we want the resulting file to be as small as possible, we save only the information crucial for the purposes of rendering. Following, we provide a list of contents of a cube file:
\begin{itemize}
	\item \emph{version}
	\item \emph{moment flag} --- a flag signifying that the cube is based on trigonometric moments. We extend the sigmoid cube structure in a similar manner with a sigmoid flag for an easier cube recognition in a rendering software.
	\item \emph{dimension} --- the number of lattice points per axis
	\item \emph{illuminant} under which the cube was fitted
	\item \emph{fitting threshold}
	\item \emph{spectral range} --- the sampling information for internal representation of spectra
	\item for every point, we store:
	\begin{itemize}
		\item \emph{coefficient representations}, along with their \emph{entry IDs} and their \emph{sizes}
		\item \emph{lattice RGB}
		\item \emph{fitting distance} --- the distance between lattice and target RGB
	\end{itemize}
\end{itemize}

Note that storing the target RGB of lattice points is unnecessary, as it can computed from the cube's dimension parameter. Although we could similarly compute the lattice RGB from the coefficient representations, we store it in order for the moment cube structure to remain compatible with the sigmoid cube structure. 

In addition to storing the cube, we extend Borgtool with the functionality to load such a cube and utilize it for either the purposes of rainbow texture uplifting (specifically, uplifting of the texture in~\cref{fig:sigmoidTexture}) or for its 3D visualization.

\section{Renderer integration} \label{sec:rendererIntegration}

In order to demonstrate the proper utilization and, subsequently, the performance of the trigonometric moment cube, we integrate it into an existing renderer --- specifically, ART. As ART already has the support for uplifting with the sigmoid cube, we solely extend both its cube structure and uplifting capabilities in a similar manner as in the Borgtool. For the scene description files, we add an option for specifying the cube file to be used for uplifting --- due to our extension in terms of the \emph{flag} parameter, we are capable of recognizing the type of the cube without it being manually specified by the user.

The uplifting process itself must be, as already concluded in this thesis, based on the trilinear interpolation of spectra at the corners of the voxel that the desired RGB falls into. Therefore, we proceed as follows:

Firstly, from the notion of the desired RGB, we obtain the 8 voxel corners along with their distances to the RGB triplet. These will later be used as weights for interpolation. We then examine the sets of entry IDs at the voxel corners and find their intersection $S$.

If $S$ is not empty, it must contain precisely one ID, and that is the ID of the constraint with which the voxel was originally seeded. To therefore reconstruct this constraint, we use only the coefficient representations corresponding to the common ID for spectral reconstruction of each voxel corner. We then carry out a weighted trilinear interpolation of the reconstructed curves, which results in the final, uplifted spectrum.

If, on the other hand, $S$ is empty, we perform the interpolation of the metamers at each of the 8 voxel corners. The resulting metamers are then passed as an input to the voxel's trilinear interpolation. The reason behind this is the same as for the coefficient recalculation (see~\cref{ssec:cubeImprovement}), and that is smooth color transitions between various metameric families under different illuminants.
\chapter{Results} \label{chap:results}

In this chapter, we talk about the results. We start of by talking about the results achieved with different parameters during implementation and which we decided to use in the end. We then proceed to evaluate

If we were to assume a single-color image map obtained by spectrally rendering one atlas entry and we were to uplift a pixel of this image map, it would be crucial for its position in the RGB cube to be the closest to a lattice point seeded by the original atlas entry, so as to utilize it during uplifting. Even a slight change in the RGB value of the image map may cause the uplifting to consider different point as primary during interpolation, which could lead to undesired behavior. Additionaly, especially in the cases of a low voxel size (i.e. high cube \texttt{dimension} parameter), such a change could even cause the pixel of the image map to be uplifted in a different voxel than the original atlas entry, which might possibly not utilize the atlas entry at all. As the variance in the RGB value of the resulting image map is inevitable due to the stochastic nature of the spectral rendering process, - navyse daco s tymto spravit

\section{Implementation parameters}

\subsection{Storing moments} \label{sec:storingMoments}

The first thing we analyze and decide on is the technique used for mapping wavelengths to a signal for the storage and subsequent reconstruction of moments. As already mentioned in~\cref{par:spectrumToCoefficientConversion}, we have the choice of both \emph{mirroring} and \emph{warping} the signal, which overall creates four options --- using only mirroring, using only warping, using both or using neither, i.e. utilizing the original signal.

Note that, although we aim for the highest possible precision in terms of curve reconstruction for the atlas lattice points, so as to lose as little information about the original atlas entry as possible, this is not the case for regular lattice points. On the contrary --- as they do not have a prior atlas entry that their spectra must approximate, we mainly aim for their smoothness in order to prevent color artifacts upon interpolation.

Additionally, when choosing the correct wavelength mapping technique, we must also take the performance of the optimizer under it into account.

We start by focusing on the accuracy of reconstruction for the atlas lattice points. We run an experiment across multiple color atlases (such as the Pantone atlas, Munsell Book of Colors or RAL atlas) and multiple illuminants in which we compare the original color of spectral curve under an illuminant with the color of a spectrum obtained by reconstruction from the original curve's coefficients under said illuminant. We then compute the average and maximum obtained color difference. For these purposes, we use the simple Delta E error due to its continuous nature.

In~\cref{sec:completeMomentError}, we provide all results obtained from these experiments. Note that using $n$ moments requires storing $n+1$ values in case mirroring is used (i.e. the moments are real) and $2n+1$ values otherwise (i.e. the moments are complex). As we are interested in the number of $coefficients$ needed for storage (and for passing to the optimizer) rather than the number of moments, we surmise the contents of~\cref{sec:completeMomentError} in~\cref{table:comparisonMomentTechnique}, where we present the errors according to the number of coefficients.

\begin{table}[t]
	\centering
	\begin{tabular}{crrrrrrrr}
		\toprule
		\multirow{4}{*}{Coefficients} &
		\multicolumn{8}{c}{Methods} \\
		\cmidrule(lr){2-9}
		&\multicolumn{2}{c}{M\&W} &
		\multicolumn{2}{c}{M\&nW} &
		\multicolumn{2}{c}{nM\&W} &
		\multicolumn{2}{c}{nM\&nW}\\
		\cmidrule(lr){2-9}
		& Avg & Max & Avg & Max & Avg & Max & Avg & Max \\
		\cmidrule(lr){1-9}
		1&24.37&202.43&24.21&202.57&24.27&202.47&24.21&202.57\\
		2&12.77&151.47&16.77&162.39&\textemdash&\textemdash&\textemdash&\textemdash\\
		3&1.71&28.28&10.23&115.27&4.03&72.86&8.39&101.6\\
		4&0.96&13.66&6.11&93.14&\textemdash&\textemdash&\textemdash&\textemdash\\
		5&0.65&10.51&2.52&38.34&1.3&28.31&2.65&30.48\\
		6&0.47&7.13&1.44&10.4&\textemdash&\textemdash&\textemdash&\textemdash\\
		7&0.43&5.7&0.97&10.47&0.75&10.81&1.1&9.58\\
		8&0.38&5.29&0.85&9.8&\textemdash&\textemdash&\textemdash&\textemdash\\
		9&0.37&5.02&0.69&6.55&0.64&7.06&0.58&5.97\\ 
		10&0.25&4.48&0.51&6.2&\textemdash&\textemdash&\textemdash&\textemdash\\
		11&0.25&4.28&0.34&4.21&0.49&4.62&0.49&5.23\\
		12&0.19&4.09&0.27&4.15&\textemdash&\textemdash&\textemdash&\textemdash\\
		13&0.19&3.92&0.26&3.83&0.31&2.97&0.37&4.88\\
		14&0.19&3.66&0.25&3.75&\textemdash&\textemdash&\textemdash&\textemdash\\
		15&0.19&3.45&0.23&3.58&0.3&2.98&0.34&3.89\\
		16&0.16&3.18&0.2&2.94&\textemdash&\textemdash&\textemdash&\textemdash\\
		17&0.16&3.07&0.19&2.32&0.28&2.49&0.3&2.6\\
		18&0.16&2.94&0.16&1.78&\textemdash&\textemdash&\textemdash&\textemdash\\
		19&0.16&2.84&0.15&1.94&0.26&2.83&0.22&2.65\\
		20&0.16&2.72&0.15&1.76&\textemdash&\textemdash&\textemdash&\textemdash\\
		21&0.16&2.6&0.15&2.0&0.26&2.65&0.22&2.75\\
		\bottomrule
	\end{tabular}
	\caption{The average and maximum \emph{Delta E} error originating from round-trips under multiple illuminants. $M$ represents mirroring, $W$ warping, and the symbol $n$ stands for their negation.}
	\label{table:comparisonMomentTechnique}
\end{table}

By observing~\cref{table:comparisonMomentTechnique}, we conclude that it is beneficial to mirror the signal. Although the non-mirroring technique performs slightly better for $c \le 9$, we are looking for a technique with which to save even the most complicated spectra which requires a lot more coefficients, and for that, non-mirroring is insufficient.

By applying the same principle when choosing whether to warp the signal, we assume that warping should provide better results. However, as the results from~\cref{table:comparisonMomentTechnique} are not entirely clear on this matter, as for $c \ge 16$, the maximum error of non-warping is lower, we put this theory to practice and test the performance of the optimizer with both warping and non-warping. 

To achieve a certain color error upon reconstruction, the warping technique usually requires less coefficients. To give an example, if we require Delta E to be $\Delta E_{ab}^*=0.1$ under the FL11 illuminant, the Munsell Book of Colours requires 14 coefficients on average with warping and 18 if not.

Even despite the benefit of having less coefficients to optimize, the average distance between the original and the fitted curves when using either of the cost functions (presented in~\cref{ssec:costFunctions}) is higher for warping. This does necessarily need to imply failure --- as the edges of the curve do not often bear significant color information, we do not require them to be reconstructed as precisely. However, occasionally, the optimizer fails in so as the fitted curve does not follow the original at all but rather ends up as a spiky spectrum susceptible to metameric artifacts. We show some of these results in~\cref{fig:warping_atlasLatticePoints}, where we compare them to the results with the non-warping technique.

\begin{figure}[t]
	\centering
	\captionsetup[subfigure]{font=footnotesize,labelfont=footnotesize}
	\captionsetup[subfigure]{justification=centering}
	\begin{subfigure}[t]{0.38\textwidth}
		\includegraphics[width=\linewidth]{img/resultsTechniqueOpt_legend.png}
	\end{subfigure} \\
	\begin{subfigure}[t]{0.32\textwidth}
		\includegraphics[width=\linewidth]{img/results_warping_orange.png}
		\caption{``orange'' patch}
		\label{fig:warping_alp_neutral50}
	\end{subfigure} \hspace{0.1em}
	\begin{subfigure}[t]{0.32\textwidth}
	\includegraphics[width=\linewidth]{img/results_warping_red.png}
	\caption{``red'' patch}
	\label{fig:warping_alp_red}
	\end{subfigure} \hspace{0.1em}
	\begin{subfigure}[t]{0.32\textwidth}
		\includegraphics[width=\linewidth]{img/results_warping_black.png}
		\caption{``black'' patch}
		\label{fig:warping_alp_black}
	\end{subfigure}
	\caption{Failure of warping when fitting atlas lattice points, shown on patches from the Macbeth Color Chart}
	\label{fig:warping_atlasLatticePoints}
\end{figure}

For the regular lattice points, we present a similar comparison in~\cref{fig:warping_regularPoints}, where we seed cubes of different sizes with distinct atlases and analyze the spectra at specific points. Not warping the signal is superior to warping even in this case, as it creates smoother spectra with less sharp edges. 

As we eventually decide on using only 3 coefficients per regular lattice points (see~\cref{ssec:noOfMoments}), and for that purpose, even warping works reasonably well, we proba

Although we eventually decide on using only 3 coefficients per regular lattice points, and for that purpose, even warping performs reasonably well, we conclude that not warping the signal is by far superior.

\begin{figure}[t]
	\centering
	\captionsetup[subfigure]{font=footnotesize,labelfont=footnotesize}
	\captionsetup[subfigure]{justification=centering}
	\begin{subfigure}[t]{0.38\textwidth}
		\includegraphics[width=\linewidth]{img/resultsTechniqueOpt_legend.png}
	\end{subfigure} \\
	\begin{subfigure}[t]{0.31\textwidth}
		\includegraphics[width=\linewidth]{img/resultsTechniqueOpt_m3_cd64.png}
		\caption{$c=3, cd=64$,\\$RGB=(222.6, 230.7, 230.7)$,\\initial atlas = Macbeth Color Chart}
		\label{fig:warping_regularPointst_m3_cd64}
	\end{subfigure}
	\begin{subfigure}[t]{0.31\textwidth}
		\includegraphics[width=\linewidth]{img/resultsTechniqueOpt_m5_cd32.png}
		\caption{$c=5, cd=32$,\\$RGB=(82.26, 172.74, 255)$,\\initial atlas = Page 14 from Munsell Book of Color}
		\label{fig:warping_regularPoints_m5_cd32}
	\end{subfigure}
	\begin{subfigure}[t]{0.31\textwidth}
		\includegraphics[width=\linewidth]{img/resultsTechniqueOpt_m7_cd16.png}
		\caption{$c=7, cd=16$,\\$RGB=(102, 17, 153)$,\\ fitted from middle}
		\label{fig:warping_regularPoints_m7_cd16}
	\end{subfigure} 
	\caption{Comparison of warping and non-warping when used for fitting regular atlas entries. Note that the figures are illustrative, as they were created with an older version of the cube and therefore may not correspond to the current results.}
	\label{fig:warping_regularPoints}
\end{figure}

We accredit the shortcomings of warping to its fixation on proper reconstruction of the middle part of the curve and its neglect of the edges. Since the 


 Since warping focuses on the more important regions of the spectrum in terms of color perception (i.e. around 550nm), it reconstructs the slight waves in this area quite precisely while neglecting the edges. Therefore, 

 Non-warping, on the other hand, focuses on the spectra as a whole, which results in approximating the shape over the whole wavelength range but not in an exact replication of any specific spikes.

We conclude that our theory is correct --- warping indeed amplifies the slight differences around the middle of the curve in order to achieve the correct color, so much that it creates spikier spectra the more the cube grows. Although this does not necessarily render the cubes created with warped signal useless, it is apparent they are prone to creating memateric artifacts such as the ones presented in~\cref{fig:metamerism}. Additionally, as we already mentioned in ref, the interpolation phase of the rendering pipeline benefits from smooth, non-spiky spectra, which is a criterion the warped spectra do not satisfy. Therefore, we it is save to discard the option of warping.






\begin{figure}[t]
	\centering
	\captionsetup[subfigure]{font=footnotesize,labelfont=footnotesize}
	\captionsetup[subfigure]{justification=centering}
	\begin{subfigure}[t]{0.60\textwidth}
		\includegraphics[width=\linewidth]{img/results_techniqueLegend.png}
	\end{subfigure} \\
	\begin{subfigure}[t]{0.45\textwidth}
		\includegraphics[width=\linewidth]{img/results_techniqueNeutral5.png}
		\caption{``neutral 5'' patch}
		\label{fig:resultsTechnique_neutral5}
	\end{subfigure} \hspace{0.1em}
	\begin{subfigure}[t]{0.45\textwidth}
		\includegraphics[width=\linewidth]{img/results_techniqueGreen.png}
		\caption{``green'' patch}
		\label{fig:resultsTechnique_green}
	\end{subfigure} \hspace{0.1em}
	\vspace{0.5em}\\
	\begin{subfigure}[t]{0.45\textwidth}
		\includegraphics[width=\linewidth]{img/results_techniqueBlueFlower.png}
		\caption{``blue flower'' patch}
		\label{fig:resultsTechnique_blueFlower}
	\end{subfigure} \hspace{0.1em}
	\begin{subfigure}[t]{0.45\textwidth}
		\includegraphics[width=\linewidth]{img/results_techniqueFoliage.png}
		\caption{``foliage'' patch}
		\label{fig:resultsTechnique_foliage}
	\end{subfigure}
	\caption{Comparison between the warped and non-warped reconstructed signal shown on multiple patches of the Macbeth Color Chart}
	\label{fig:resultsTechniques}
\end{figure}


Although the shortcomings of warping can already be perceived in e.g.~\cref{fig:resultsTechnique_foliage} or~\cref{fig:resultsTechnique_blueFlower}, the failure in the reconstruction of the curve's edges does not have a significant effect of the the resulting RGB color, as the source of color is mainly focused around the middle of the curve. However, if the edges are extremely distinct from the rest of the curve (see~\cref{fig:resultWorstWarp}), they tend to provide unanticipated color information. In such cases, warping the signal presents a disadvantage.

Obviously, the Delta E error caused by unnecessary warping can be reduced to almost 0 by passing the computed coefficients to the optimizer, which then alters the curve so that it evaluates to the correct RGB. However, because we lose the notion of the curve's desired shape and because warping does not focus on the edges, the optimizer is apt to amplify the already existing slight bumps in the middle. This behavior may therefore cause the resulting shape to be extremely distinct from the desired one. 

The non-warping technique, shown in~\cref{fig:resultWorstNonWarp}, is not susceptible to this kind of behavior. Although it creates a rather significant Delta E error, we can observe that the shape of the reconstructed spectrum roughly resembles the original shape. Such a behavior is desired in our case, as the reconstructed reflectance is less prone to cause metameric artifacts under different illuminants. Additionally, as the non-warping technique forces the optimizer to not prioritize specific parts of the curve, the optimization is prone to slightly altering the shape as a whole rather than creating 
irregularities in the middle. Therefore, regardless of the average error, we assume the non-warping technique to outperform warping both when performing simple round-trips, but also if we use this method during the optimization process of fitting the cube.



Obviously, the ideal solution would be to store the spectra with the method that provides better Delta E error and use non-warping for cube fitting afterwards. However, such an approach is impractical. Firstly, currently, as the first step of the spectral reconstruction is the conversion of wavelength array to a phase signal, using only one method means we can save this signal prior to fitting and reuse it, thus lowering time complexity. Using both methods would require either storing two phase signals, or recomputing them during each reconstruction. Secondly, as we use non-warping for cube fitting anyway, saving only some atlas entries with warping would be both impractical and would not provide too many benefits. Therefore, we leave the possibility of implementation of the support of both methods as future work.

We decide not to use one technique for one thing. That's too much work.

Ku cost:

When fitting other spectral data, the optimizer does not behave as drastically as shown in~\cref{fig:resultsCostFunctions}. On the contrary, many fitted altas entries (such as the ``blue flower'' of the Macbeth Color Checker) resemble their input quite well. However, we must focus on the worst-case scenario as we do not wish to experience any metameric artifacts, not even in one color.

The heuristic regarding the threshold does not, on average, need to be invoked more than 2-3 times when fitting an arbitrary atlas to a sufficiently-sized cube (i.e. 64-dimensional). It is clear that by lowering the cube's dimension, the heuristic becomes more utilized. For example, for an 8-dimensional cube, it reaches up to 100 invocations, and even then the fitting may not always be successful.

Our cost functions give us the ability to control the shape of the resulting spectra. By further utilizing them, we could fit the neighbors of atlas lattice points so that their spectra is extremely similar. Applying this method to the whole cube-fitting process could therefore create an uplifting model with a rather uniform spectra. Such an uplifting model is especially desired for the interpolation phase in rendering.

However, two problems already arise with this approach. Firstly, by growing the cube from multiple atlas entries at the same time, there is bound to be a point in which neighbors are fitted from different prior atlas entries. By our premise, this would mean that the spectra of these two points could be vastly different.

Another issue is the significant increase in time complexity. Even excluding the complex calculations the optimizer must perform during minimization, the computation of around 400 residuals takes a lot more time than just computing the original 3. The process of fitting a 32-dimensional cube, regardless of the number of its moments and the allowed optimizer's threshold, then takes hours instead of mere minutes when executed on an ordinary desktop PC. This renders our cost functions unusable for cube fitting and we must therefore settle for using the RGB cost functions only.

Therefore, we use the original RGB difference cost functions for fitting regular point and our new cost functions for fitting atlas lattice point.

\subsection{Number of moments} \label{ssec:noOfMoments}

\subsection{Cost functions} \label{ssec:costFunctions}

In addition to the moment storage technique, another thing greatly affecting the performance of the fitting are the cost functions of optimizer.

For the fitting of the sigmoids, Borgtool uses three cost functions, or \emph{residuals}, each of them specifying the absolute color difference in one axis of the RGB cube. Such an approach outperforms both the Euclidean color distance and even the Delta E difference --- the higher the number of meaningful residuals, the more information about the coefficients' behavior can the optimizer deduce, which, in turn, results in faster and more precise convergence to global optimum.

As this approach performs rather decently in terms of both time complexity and the obtained results for the sigmoids, we try it out for the purposes of our optimization as well. 

In case of fitting of the regular lattice points (see~\cref{ssec:cubeFitting}), which requires the manipulation of 3 coefficients, the obtained results are satisfactory, both for the fitting in the second round and in the latter rounds (see~\cref{fig:costFunctionsRegularFitting}). The resulting curves are smooth, which suits the interpolation process during rendering, they evaluate to the desired RGB values, and the run-time is even better than for the sigmoids. Therefore, we utilize this approach for the regular lattice points.

\begin{figure}[t]
	\centering
	\begin{subfigure}[t]{0.4\textwidth}
		\includegraphics[width=\linewidth]{img/cost_functions_regular_legend.png}
	\end{subfigure} \\
	\begin{subfigure}[t]{0.45\textwidth}
	\includegraphics[width=\linewidth]{img/cost_functions_regular_round2.png}
	\caption{Fitting in the second round, i.e. the prior coefficients are the result of ``recomputation'' of the fitted coefficients of an atlas lattice point}
	\label{fig:costFunctionsRegularRound2}
	\end{subfigure} \hspace{0.1em}
	\begin{subfigure}[t]{0.45\textwidth}
		\includegraphics[width=\linewidth,height=0.2\textheight]{img/cost_functions_regular_round8.png}
		\caption{Fitting in round 8/20, where the prior spectrum is that of a regular lattice point}
		\label{fig:costFunctionsRegularRound8}
	\end{subfigure} 
	\caption{Fitting of regular lattice points with 3 cost functions specifying the RGB difference}
	\label{fig:costFunctionsRegularFitting}
\end{figure}

However, when fitting the atlas lattice points (see~\cref{ssec:startingPointsFitting}), the fact that the cost functions do not take the resulting shape of the spectra into account in any way works to our disadvantage. We show an example of this in~\cref{fig:resultsCostFunctions} on the magenta plot, which demonstrates the performance of fitting of atlas lattice points with RGB cost functions only. Although it terminates as successful (as the RGB of the resulting curve is within the fitting threshold of the target RGB), the reflectance curve takes on a sinusoidal shape with a rather high amplitude, therefore losing resemblance to the original curve. This is due to definition of coefficients, which are, in their nature, Fourier coefficients, and are therefore prone to exhibiting this type of behavior. This is mainly perceivable when the color distance $d$ between the atlas entry and atlas lattice point is rather high, as it gives the optimizer enough room to make visible changes in the shape of the reconstructed spectrum before converging below the fitting threshold.
 
Note that we compare the fitted spectra not with the original atlas spectra, but with the spectra that is reconstructed from the original's coefficients to keep track of the optimizer's ability to mimic its input.

\begin{figure}[t]
	\centering
	\captionsetup[subfigure]{font=footnotesize,labelfont=footnotesize}
	\captionsetup[subfigure]{justification=centering}
	\begin{subfigure}[t]{0.70\textwidth}
		\includegraphics[width=\linewidth]{img/results_costFunctions_legend.png}
	\end{subfigure} \\
	\begin{subfigure}[t]{0.45\textwidth}
		\includegraphics[width=\linewidth]{img/results_costFunctions_orange.png}
		\caption{``orange'' patch of the MCC, $c = 8$, $d = 11.84$}
		\label{fig:resultsCostFunctions_orange}
	\end{subfigure} \hspace{0.1em}
	\begin{subfigure}[t]{0.45\textwidth}
		\includegraphics[width=\linewidth]{img/results_costFunctions_mcb0706.png}
		\caption{5YR 7/6 patch of the MBC, $c = 14$, $d = 5.85$}
		\label{fig:resultsCostFunctions_mcb0706}
	\end{subfigure} 
	\vspace{0.5em}\\
	\begin{subfigure}[t]{0.45\textwidth}
		\includegraphics[width=\linewidth]{img/results_costFunctions_mcb0725.png}
		\caption{N 7.25 patch of the MBC, $c = 20$, $d = 12.67$}
		\label{fig:resultsCostFunctions_mcb0725}
	\end{subfigure} \hspace{0.1em}
	\begin{subfigure}[t]{0.45\textwidth}
		\includegraphics[width=\linewidth]{img/results_costFunctions_darkskin.png}
		\caption{``dark skin'' patch of the MCC, $c = 12$, $d = 11.55$}
		\label{fig:resultsCostFunctions_darkskin}
	\end{subfigure}
	\caption{Comparison between the RGB cost functions and our method for fitting atlas lattice points}
	\label{fig:resultsCostFunctions}
\end{figure}

We therefore conclude that we must incorporate the requirement of curve shape similarity into our cost functions. Following, we review the cost functions that we tried along with their performance.

Firstly, we implemented an approach similar to that for determining the number of coefficients with which to store atlas entries (see ref), i.e. we utilized the color error under a fluorescent illuminant (specifically, FL11). We defined three additional cost functions, each specifying the difference between the original and the reconstructed curve's RGB under the FL11 illuminant in one of the axes. If their values fell below a specific threshold, we terminated the fitting process as successful, if not, we increased the threshold and tried again.

Although this approach was successful on average (the average threshold was around $t = 0.025$), on occasion, the threshold often needed to be increased to values so high that it became obsolete. Using only one error, either the Euclidean distance or the Delta E error, caused similar issues.

Therefore, we decided to focus on the actual distance between the two curves. Our first attempt consisted of defining one residual per wavelength sample specifying the absolute distance between the two spectra at said wavelength (as the least square error proved to perform worse). We examined the performance of both around 360 residuals (i.e. 1nm increment between samples) and 36 residuals (10nm increment, as defined in most color atlases) when used alongside the 3 already defined RGB residuals. For $c \le 9$, both of these options performed reasonably well. When fitting more coefficients, the sufficient threshold was, on average, around $t = 0.0096$, both before and after the introduction of our improvement of optimizing only first 4 coefficients, and although its maximum ended up being $t = 0.23$, overall, this method outperformed the previous one.

As we suspected that the failures were caused by the abundance of cost functions, we summed up their values and saved them into a single residual, which, when divided by the number of spectral samples, represented the average error per sample.

As the importance of curve samples in terms of proper color reconstruction is mainly placed on their middle (at around 550nm), we attempted to add a heuristic-based weighting factor in an effort to focus on minimizing the distance between the two curve there. However, we did not succeed in improving our results, and we therefore dropped the experiment and examined the performance without weighting the values.

Such an approach substantially outperformed the previous ones, and even more after we decided to specify the absolute error rather than the least square error. The average error ended up being only about $d = $, and, of yet, no entry requiring $d > $ has been encountered. Therefore, we concluded that the optimal solution to our problem is to set 4 residuals, 3 specifying the RGB difference, and 1 specifying the average distance per sample.

We present some of the results achieved with our cost functions on the green plot in~\cref{fig:resultsCostFunctions}, where we compare them to fitting with RGB cost functions only. In~\cref{fig:resultsCostFunctions_mcb0725} and~\cref{fig:resultsCostFunctions_darkskin}, we specifically focus on the most problematic spectra with the highest distance $d$ between the atlas lattice point and the atlas entry.

Although our approach substantially reduces the appearance of sinosidual-like behavior, it does not diminish it completely. It can be observed especially for atlas lattice points with $c > 14$ (see~\cref{fig:resultsCostFunctions_mcb0725}). This is due to the fact that we optimize only the first 4 coefficients, which, in their nature, are prone to creating such patterns.

Another drawback of our approach is the substantially larger time complexity, especially if improvement heuristics need to be applied. By examining the scope of the optimizer and utilizing its options further, or maybe even resorting to a different optimizer, we might be able to improve upon both the time complexity, and the resulting spectral shape.

However, as the runtime is not the focus of this thesis and as the resulting shapes are satisfactory for the purposes of this these, we focused our efforts elsewhere and leave these improvements for future work.

 However, as that is not the focus of this thesis, we leave the improvements in time complexity for further work.


\subsection{Number of moments}
 
Maybe add a section about unfittable cubes?

However, correct round-trips are not the only factor we need to take into consideration when fitting the cube. We also need to focus on both the \emph{smoothness of the resulting spectra} and the \emph{behavior of the optimizer} under the current technique.

The smoothness of the spectra is especially important for the interpolation process that takes place during the rendering. Interpolating multiple spiky, non-similar spectra would result in similarly uneven spectra, which, in addition to incorrect color, may be susceptible to extreme metameric artifacts.  

The behavior of the optimizer also plays a big role. During the optimization, it takes into account only the resulting RGB of the curve rather than the shape of the curve itself. Therefore, it does not aim for a curve with a similar shape than its neighbor, which may likewise cause issues during the interpolation.

When we fit from the middle, we only aim for the smoothness of the resulting spectra and for the behavior of the optimizer. We therefore want as less coefficients as possible and, as we do not care about round trips, we can use any of the techniques available.

We already said that using 9 coefficients is unecessary, as 3 already create rather smooth spectra that is way better for interpolation. However, we here find out that it is not only unnecessary but extremely discouraged. We can see in the image how the fitting proceeds. It amplifies the already existing wave-like patterns, exhibiting the same behavior as it did during atlas fitting, which is caused by the cost functions not considering the shape of the resulting spectra. By allowing only three coefficients, we limit the optimizer so it must create smooth spectra and it cannot create crazy shapes. Additionaly, it is better for performance - fiting of a 9-coefficients takes around blabla, while fitting 3 takes blabla. We provide measurements here? of performance

Although we cannot control shape, we can control smoothness. We want smooth spectra both for interpolation purposes and because smooth spectra is less prone to metameric artifacts, see chapter 1.

The behavior we talked about before of the optimizer is a problem 
It would be extremely beneficial for the

For middle fitting, we always recommend 2 moments, 3 coeffs due to the runtime. Moreover, they are very smooth but not straight which is ideal for our purposes. Obviously, we can fit with higher but that takes a lot of time and does not provide any advantage. The default setting is therefore 3 coefs, 2 moments.

Obviously, fitting with these causes metameric artifacts, similar to the ones created by sigmoid, we show these in PICTURE. We can also show metameric artifacts when fitting with more? (try this maybe)

Another slight trick is to use a lower number of coefficients - we therefore do not get as big artifacts because we cannot possibly reconstruct such crazy spectra with low number of coefficients. 

The optimizer is always faster for lower number of coefficients as it does not need to change them up as much. Also faster but does not change them so much SO it results in spectrum that is similar to the first one. However, we cannot possibly simulate the curve with only the limited number of coefficients, the table says so. Therefore we must find something in the middle. We try not to focus on performance here because obviously, fitting anything that is higher than 32 takes hours and we want to have it look the best way we can.

\section{Performace}

heuristics performance, overall runtime of the cube

\section{Rendering}

- which technique gives the best results (metamerism results)


Also mention that it is multi-threaded and performance is not really a priority - the cube has to be created only once and then can be reused as much as the artists need.


\chapter*{Conclusion}
\addcontentsline{toc}{chapter}{Conclusion}

In this thesis, we presented the first method capable of constraining the spectral uplifting process with an arbitrary set of target spectra. By utilizing a trigonometric moment-based approach for spectral representation, the RGB values of the target spectra are accurately uplifted to their original spectral shapes, while the rest of the RGB gamut uplifts to smooth spectra. This results in smooth transitions between
the various metameric families that originate from the constraining process.

Our model shows a slight weakness when uplifting very dark colors, which we attribute both to the inability of the moment-based spectral representations to represent constant spectra, and to the deficiency of the optimization process used during the creation of our uplifting model. However, even including these minor drawbacks, the results in terms of color accuracy are noteworthy, as the uplifted curves describe the original ones with negligible differences.

Neither the memory, nor the execution time of either the creation or the utilization of our model are optimal: the new and so far unique capability to perform targeted uplifts comes at the cost of some overhead that is not present in e.g. the unconstrained sigmoid uplift technique of~\citet{upsamplingJakobHanika}.

In the future, we will primarily focus on utilizing a more suitable and memory efficient structure for storing the constraints, such as a kD-tree or an octree. Secondly, we will improve the execution time by optimizing the moment-based spectral reconstruction process, and, furthermore, we will focus on improving other minor deficiencies, reviewed in~\cref{sec:futureWork}.


%%% Bibliography
\include{bibliography}

%%% Figures used in the thesis (consider if this is needed)
%%\listoffigures

%%% Tables used in the thesis (consider if this is needed)
%%% In mathematical theses, it could be better to move the list of tables to the beginning of the thesis.
%%\listoftables

%%% Abbreviations used in the thesis, if any, including their explanation
%%% In mathematical theses, it could be better to move the list of abbreviations to the beginning of the thesis.
%%\chapwithtoc{List of Abbreviations}

%%% Attachments to the master thesis, if any. Each attachment must be
%%% referred to at least once from the text of the thesis. Attachments
%%% are numbered.
%%%
%%% The printed version should preferably contain attachments, which can be
%%% read (additional tables and charts, supplementary text, examples of
%%% program output, etc.). The electronic version is more suited for attachments
%%% which will likely be used in an electronic form rather than read (program
%%% source code, data files, interactive charts, etc.). Electronic attachments
%%% should be uploaded to SIS and optionally also included in the thesis on a~CD/DVD.
%%% Allowed file formats are specified in provision of the rector no. 72/2017.
\appendix
\chapter{Software user guide}

In this appendix, we provide a user guide for compiling and running both Borgtool (for the purposes of creating the cube) and ART (for the purposes of its utilization).

Both softwares have been tested on Ubuntu~20.04 with CMake~3.16.3 and g++~8.4.0. As we do not provide any binaries, the user must compile the projects first.

\section{Borgtool}

To build Borgtool, execute the following steps:
\begin{enumerate}
	\item Unzip the provided attachment and open the $Borgtool$ folder 
	\item Make sure you have installed ART and Ceres (see CMakeLists.txt for more details about necessary dependencies)
	\item Build the project using CMake
\end{enumerate}

It is possible to run the program with several options. Following, we provide the list of the ones supported with the trigonometric moment-based method:

\begin{verbatim}
Usage: borgtool [-options]
where the options include:
    (-cc | --createCube) <filename>
        fit new RGB coefficient cube
    -mo | --momentOpt 
        use moments for spectral representation
    (-cd | --cubeDimension) <numpoints>
        # of cube lattice points in one dimension
    (-ft | --fittingThreshold) <x>
        threshold in fraction 1/x of voxel size
    -pi | --progressImages
        generate fitting progress images
    -ofr | --onlyFirstRound
        do not attempt to improve the cube
    (-tex | --texture) <filename>
        create EXR test texture for cube testing
    (-ctx | --cubeTexture) <filename>
        create EXR test texture with cube data
    (-tv | --textureV) <0..1>
        HSV V value for the two preceding textures
    (-sfd | --showFittingDelta) <0..1>
        show delta between lattice and target RGB
    (-uc | --useCorner) <index>
        use coefficents from voxel corner #
    (-ply | --generatePLY) <filename>
        create PLY geometry for UCC file
    -srgb | --sRGB 
        use sRGB (default)
    (-a | --atlas) <atlasID>
        seed the cube with atlas with ID #
\end{verbatim}

Examples of usage: 
\begin{itemize}
	\item \texttt{borgtool -mo -cc cube\_mcob -cd 64 -a 0}
	
	This command creates a new trigonometric moment-based RGB cube of size $64^3$ seeded with an atlas with $ID=0$, which is the Munsell Book of Color.
	
	\item \texttt{borgtool -mo -cc cube\_middle}
	
	This command creates a new trigonometric moment-based RGB cube of size $32^3$. As no atlas is specified, the cube grows from its center.
	
	\item \texttt{borgtool -mo -rc cube\_mcc -ctx test\_texture -tv 0.5}
	
	This command loads a cube with the name \texttt{cube\_mcc} and utilizes it for uplifting a rainbow texture (see~\cref{fig:uplift_colourful_texture}). The brightness of the texture is set to $0.5$, and the resulting uplift is stored in a file named \texttt{test\_texture}.

\end{itemize}

\section{ART}
To build ART, execute the following steps:
\begin{enumerate}
	\item Unzip the provided attachment and open the $ART$ folder 
	\item Build the project using the instructions from the ART website at\newline \url{https://cgg.mff.cuni.cz/ART/download/}
\end{enumerate} 

In order to utilize the cubes created by Borgtool, execute the following steps:
\begin{enumerate}
	\item Copy the cubes you wish to utilize to \texttt{ART/ART\_Resources/SpectralUplift}
	\item Add the following code snippet:
	\begin{verbatim}
        SET_UPLIFT_CUBE(
            "my_cube_name.ucc"
        ),
	\end{verbatim}
	to the action sequence of the \texttt{.arm} file whose image map you wish to uplift. If this option is omitted, the sigmoid cube is utilized.
	\item run the \texttt{artist} command with the desired parameters
\end{enumerate}

\subsection{Example scenes}

In the \texttt{Resources} folder found in the attachment of this thesis, we provide multiple example scenes along with textures and cubes, some of which have been used to render images in this thesis. Following, we provide instructions on how to replicate these renders, and overview the contents of the attached folder.

Before running the example scenes, do the following:
\begin{enumerate}
	\item Copy all the \texttt{.ucc} files from the provided \texttt{Resources/cubes} folder to\newline \texttt{ART/ART\_Resources/SpectralUplift}
	
	\item Copy all the \texttt{.tif} and \texttt{.tiff} files from the provided \texttt{Resources/textures} folder to \texttt{ART/ART\_Resources} 
\end{enumerate}

Contents of the \texttt{Resources} folder include:
\begin{itemize}
	
	\item Resources for replicating~\cref{fig:results_art_scene}:
	\begin{itemize}
		\item \texttt{scenes/Three\_Pages\_Original.arm} file for replicating the original spectral render. Rendering it does not require any additional resources.
		\item \texttt{scenes/Three\_Pages\_Texture.arm} file for replicating the constrained and the sigmoid-based uplift. 
		Additional resources required are as follows:
		\begin{itemize}
			\item uplift cube: \texttt{cubes/three\_pages.ucc}
			\item image maps:
			\begin{itemize}
				\item[] \texttt{cubes/three\_pages\_page10.tiff}
				\item[] \texttt{cubes/three\_pages\_page12.tiff}
				\item[] \texttt{cubes/three\_pages\_page14.tiff}
			\end{itemize}
		\end{itemize}
	\end{itemize}

	\item Resources for replicating~\cref{fig:colorimetric_properties}:
	\begin{itemize}
		\item scene file: \texttt{Gradient\_Texture.arm}
		\item image map: \texttt{textures/gradient.tif}
		\item cubes: 
		\begin{itemize}
			\item[] \texttt{cubes/mboc\_cd32.ucc}
			\item[] \texttt{cubes/none\_cd32.ucc}
			\item[] \texttt{cubes/pantone\_cd32.ucc}
		\end{itemize}
	\end{itemize}
	By default, the cube specified for uplifting in the \texttt{Gradient\_Texture.arm} scene is \texttt{pantone\_cd32}. In order to render with a different cube, change the uplift cube in the scene description.
	
	\item Three additional scenes uplifting three pages of the Munsell Book of Color:
	\begin{itemize}
		\item \texttt{scenes/Page10\_Texture.arm}
		\item \texttt{scenes/Page12\_Texture.arm}
		\item \texttt{scenes/Page14\_Texture.arm}
	\end{itemize}
	For each of the scenes, a corresponding image map and an uplift cube can be found in \texttt{Resources/textures} and \texttt{Resources/cubes} respectively.
\end{itemize}
\chapter{Attachments}

\section{Delta E error caused by moment sampling} \label{sec:completeMomentError}

\begin{table}[h]
	\centering
	\begin{tabular}{crrrrrrrr}
		\toprule
		\multirow{4}{*}{Moments} &
		\multicolumn{8}{c}{Methods} \\
		\cmidrule(lr){2-9}
		&\multicolumn{2}{c}{M\&W} &
		\multicolumn{2}{c}{M\&nonW} &
		\multicolumn{2}{c}{nMW} &
		\multicolumn{2}{c}{nMnW}\\
		\cmidrule(lr){2-9}
		& Avg & Max & Avg & Max & Avg & Max & Avg & Max \\
		\cmidrule(lr){1-9}
		0&24.37&202.43&24.21&202.57&24.27&202.47&24.21&202.57\\
		1&12.77&151.47&16.77&162.39&4.03&72.86&8.39&101.6\\
		2&1.71&28.28&10.23&115.27&1.3&28.31&2.65&30.48\\
		3&0.96&13.66&6.11&93.14&0.75&10.81&1.1&9.58\\
		4&0.65&10.51&2.52&38.34&0.64&7.06&0.58&5.97\\
		5&0.47&7.13&1.44&10.4&0.49&4.62&0.49&5.23\\
		6&0.43&5.7&0.97&10.47&0.31&2.97&0.37&4.88\\
		7&0.38&5.29&0.85&9.8&0.3&2.98&0.34&3.89\\
		8&0.37&5.02&0.69&6.55&0.28&2.49&0.3&2.6\\ 
		9&0.25&4.48&0.51&6.2&0.26&2.83&0.22&2.65\\
		10&0.25&4.28&0.34&4.21&0.26&2.65&0.22&2.75\\
		11&0.19&4.09&0.27&4.15&0.28&2.71&0.18&1.76\\
		12&0.19&3.92&0.26&3.83&0.25&2.26&0.17&1.82\\
		13&0.19&3.66&0.25&3.75&0.23&1.96&0.18&1.88\\
		14&0.19&3.45&0.23&3.58&0.24&2.11&0.17&1.85\\
		15&0.16&3.18&0.2&2.94& 0.2&3.66&0.17&1.65\\
		16&0.16&3.07&0.19&2.32&0.23&6.45&0.16&1.51\\
		17&0.16&2.94&0.16&1.78&0.28&6.68&0.16&1.59\\
		18&0.16&2.84&0.15&1.94&0.29&4.04&0.15&1.73\\
		19&0.16&2.72&0.15&1.76&0.4&5.34&0.16&1.79\\
		20&0.16&2.6&0.15&2.0&0.55&5.73&0.16&1.69\\
		21&0.16&2.48&0.15&1.49&0.58&4.83&0.16&1.68\\
		22&0.16&2.36&0.15&1.49&0.35&3.38&0.15&1.56\\
		23&0.16&2.29&0.14&1.59&0.36&2.62&0.14&1.72\\
		24&0.16&2.2&0.14&1.63&0.25&1.72&0.15&1.86\\
		25&0.16&2.18&0.14&1.7&0.24&1.44&0.16&1.89\\
		26&0.16&2.14&0.14&1.7&0.22&1.7&0.17&1.96\\
		27&0.16&2.09&0.14&1.68&0.24&1.63&0.17&2.23\\
		28&0.16&2.02&0.14&1.73&0.22&1.45&0.16&2.39\\
		29&0.17&2.44&0.13&1.69&0.18&1.5&0.15&1.54\\
		30&0.17&2.32&0.13&1.71&0.16&1.21&0.17&1.72\\
		31&0.17&2.97&0.13&1.71&0.16&1.23&0.21&2.33\\
		32&0.18&2.9&0.13&1.69&0.17&1.12&0.28&2.27\\
		33&0.2&2.64&0.13&1.69&0.14&1.18&0.26&2.47\\
		34&0.2&2.83&0.13&1.66&0.13&1.67&0.2&2.2\\
		35&0.18&2.11&0.14&1.66&0.15&2.65&0.23&1.85\\
		36&0.18&1.95&0.14&1.64&0.15&2.38&0.22&1.83\\
		37&0.16&2.05&0.14&1.64&0.18&2.64&0.15&1.45\\
		38&0.12&1.89&0.14&1.66&0.19&3.25&0.1&0.81\\
		39&0.13&1.66&0.13&1.65&0.31&3.53&0.07&0.77\\
		40&0.13&1.79&0.13&1.66&0.24&2.48&0.07&0.58\\
		\bottomrule
	\end{tabular}
	\label{table:completeMomentError}
\end{table}

\openright
\end{document}
