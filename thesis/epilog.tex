\chapter*{Conclusion}
\addcontentsline{toc}{chapter}{Conclusion}


However, it has to be noted that the approach presented in this paper works perfectly fine for target spectra sets up to several dozen, or maybe even low numbers of hundred, data points. Which is sufficient for typical usage in VFX scenarios, where only a few key assets (like e.g. the main colours of the costume of a lead character) are measured on set, in order to later constrain the spectral uplift of virtual doubles of this character.
We also note that our approach, while being both slower to fit and slower to render than the sigmoid technique, offers the unique capability of targeted uplifting, which was simply not available before. As such, we deem a somewhat slower performance compared to sigmoid-based uplift to be an acceptable price to pay.