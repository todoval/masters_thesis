\chapter*{Conclusion}
\addcontentsline{toc}{chapter}{Conclusion}

In this thesis, we presented the first method capable of constraining the spectral uplifting process with an arbitrary set of target spectra. By utilizing a trigonometric moment-based approach for spectral representation, the RGB values of the target spectra are accurately uplifted to their original spectral shapes, while the rest of the RGB gamut uplifts to smooth spectra. This results in smooth transitions between
the various metameric families that originate from the constraining process.

Our model shows a slight weakness when uplifting very dark colors, which we attribute both to the inability of the moment-based spectral representations to represent constant spectra, and to the deficiency of the optimization process used during the creation of our uplifting model. However, even including these minor drawbacks, the results in terms of color accuracy are noteworthy, as the uplifted curves describe the original ones with negligible differences.

Neither the memory, nor the execution time of either the creation or the utilization of our model are optimal: the new and so far unique capability to perform targeted uplifts comes at the cost of some overhead that is not present in e.g. the unconstrained sigmoid uplift technique of~\citet{upsamplingJakobHanika}.

In the future, we will primarily focus on utilizing a more suitable and memory efficient structure for storing the constraints, such as a kD-tree or an octree. Secondly, we will improve the execution time by optimizing the moment-based spectral reconstruction process, and, furthermore, we will focus on improving other minor deficiencies, reviewed in~\cref{sec:futureWork}.
