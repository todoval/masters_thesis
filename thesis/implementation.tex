\chapter{Implementation}

We approach the problem of spectral uplifting similarly to~\citet{upsamplingJakobHanika}, where an uplifting model is created prior to rendering. Our implementation therefore consists of two parts --- \emph{model creation} and its subsequent \emph{utilization} in a rendering software. 

For the first part, we extend an already existing uplifting tool, the \emph{Borgtool}, which is currently used for creating sigmoid-based RGB cubes as in~\cref{alg:upliftingAlgSigmoid}. We add the possibility for creating trigonometric moment-based cubes, i.e. for the spectra to be stored with trigonometric moments rather than sigmoid coefficients. We also add an option for constraining such a cube with a user-specified atlas.

We then show the performance of such a model by integrating it in ART, which, up until now, used only one built-in sigmoid-based cube for all its uplifting processes.

\section{Uplifting model}

The core of this section is the already mentioned Borgtool. It is a stand-alone, non-open source thing that was created by Weta and is .

The output of the Borgtool is an RGB cube structure which contains multiple entries in form of lattice points. Following, we name the main parameters of a single cube entry of the already supported sigmoid-based cube:
\begin{itemize}
	\item \emph{target RGB} --- the actual RGB that the point has in the cube.
	\item \emph{coefficients} --- the sigmoid coefficients used to reconstruct a spectrum so it matches the target RGB.
	\item \emph{lattice RGB} --- the actual RGB that the reconstructed spectrum evaluates to. Ideally, this should match the target RGB.
\end{itemize}
Along with its entries, the resulting cube structure also stores a few other properties, both \emph{static}, such as the illuminant according to which the RGB cube is uplifted, and \emph{user-adjustable}, such as the cube dimension or the fitting threshold (i.e. the maximum allowed difference between the target and the lattice RGB).

Our trigonometric moment-based cube can be viewed as an extension of the sigmoid cube --- in addition to the already existing parameters, we add a user-adjustable \texttt{coefficientCount} variable which specifies the number of coefficients that are to be used for most of the entries. Furthermore, for the purposes of atlas constraining, we extend the cube entry with an optional \texttt{entryThisIsBasedOn} pointer to an atlas entry. This is where the main difference between our and the sigmoid cube lies --- while the sigmoid cube regards all of its points as equal, we distinguish between \emph{atlas lattice points}, i.e. the lattice points that correspond to specific atlas entries; and \emph{regular points}, which do not. We place special emphasis on the atlas lattice points, as we require their spectra to be as precise and close to the original spectra as possible. As a result, we choose to always store such spectra with the maximum available coefficients (currently, $9$). Therefore, the \texttt{coefficientCount} variable applies to the regular points only. We explain the reasoning behind this decision and its impact on rendering and overall performance more thoroughly as we continue with this section.

Our uplifting process is also similar to the one already implemented in the Borgtool, which closely follows~\cref{alg:upliftingAlgSigmoid}. Following are the individual steps of the process:
\begin{enumerate}
	\item \emph{Initialization}
	\item \emph{Fitting of starting points}
	\item \emph{Cube fitting}
	\item \emph{Cube improvement}
	\item \emph{Cube storage}
\end{enumerate}

\subsection{Initialization}

This part of the run is responsible for three things:
\begin{itemize}
	\item parsing of the parameters
	\item initialization of the cube and its entries with default values
	\item loading of the required color atlases
\end{itemize}

The initialization of the cube is pretty straightforward, as all of its properties are either user-defined or set to default (note: the default illuminant is always D65). The number of cube entries is directly proportional to the cube's \texttt{dimension} parameter, which specifies the number of entries per one axis. This renders the total number of entries to $dimension^3$. As the lattice points are positioned evenly, their target RGB values are then equivalent to their coordinates in the RGB cube.

The loading of the atlases is a bit more complicated. Firstly, a single color atlas is inputted in a form of a simple .txt file, which contains merely a list of entries in a textual form as shown in~\cref{fig:macbethSampleText}. Therefore, it requires parsing.

\begin{figure}[t]
	\lstset{
		string=[s]{"}{"},
		comment=[l]{:},
		commentstyle=\color{black},
		basicstyle=\scriptsize
	}
	\begin{lstlisting}
Entry ID:   orange
---------------------------------------------------------------------------
Description           :  "orange" patch of the Macbeth colour checker
Type                  :  reflectance spectrum
Fluorescence data     :  no
Measurement device    :  
Measured by           :  
Measurement date      :  

Sampling information
--------------------
Type	    	      :  regular
Start                 :  380.0 nm
Increment             :  5.0 nm
Maximum sample value  :  100.0

ASCII sample data
-----------------
{6.143748,  5.192119,  4.867970,  5.092529,  4.717562,  4.663087,  4.455331,  4.562958,  4.517197,  4.536289,  4.454180,  4.543101,  4.491708, ... }

	\end{lstlisting}
	\caption{A sample entry from the Macbeth Color Checker atlas.}
	\label{fig:macbethSampleText}
\end{figure}

Moreover, the spectral data obtained from the atlases cannot be stored in the Borgtool directly, so as to avoid extreme memory requirements arising with large atlases. To solve this problem, we take advantage of the trigonometric moments.

We store the spectral curves of the individual atlas entries by using the Fourier coefficients as described in~\cref{par:spectrumToCoefficientConversion}. We use the maximum number of available moments (currently 9, see explanation in~\cref{ssec:ceresSolver}), and we mirror but do not warp the signal prior to coefficient computation. We explain the reasoning behind this in~\cref{sec:storingMoments}, where we run experiments to decide on the most efficient and precise method for storing coefficients.

\subsection{Fitting of starting points}

In order to uplift the whole cube as described in~\cref{alg:upliftingAlgSigmoid}, we must first fit one or more \emph{starting points} whose coefficients can then be used as prior for the fitting of other lattice points. For these purposes, we utilize the user-specified color atlas.

Ideally, we would assign each entry of the color atlas one lattice point and denote these points as starting points. However, as the RGB value of an atlas entry can be virtually any triplet within the (0,1) range, it is most likely that the atlas entries do not evaluate to any specific lattice RGB but rather to an RGB of a point somewhere in the middle of a cube voxel.

One way to solve such a problem would be to create a complete injective mapping between the atlas entries and their closest lattice points. However, although such an approach is correct in theory, in practice, it is prone to fail during the uplifting process. This is due to the core principle of the uplifting processof a single RGB point, which, as mentioned in ref, is a weighted interpolation of either spectra or coefficients of its neighbors in the RGB cube. If we were to assume a single-color image map obtained by spectrally rendering one atlas entry and we were to uplift a pixel of this image map, it would be crucial for its position in the RGB cube to be the closest to a lattice point seeded by the original atlas entry, so as to utilize it during uplifting. Even a slight change in the RGB value of the image map may cause the uplifting to consider different point as primary during interpolation, which could lead to undesired behavior. Additionaly, especially in the cases of a low voxel size (i.e. high cube \texttt{dimension} parameter), such a change could even cause the pixel of the image map to be uplifted in a different voxel than the original atlas entry, which might possibly not utilize the atlas entry at all. As the variance in the RGB value of the resulting image map is inevitable due to the stochastic nature of the spectral rendering process, we opt for utilizing all 8 points on the corners of the voxel instead of just the closest point. We assign the coefficients of the atlas entry to each of them. We call this process the \emph{seeding} of the cube, and we classify the seeded points as \emph{atlas lattice points}.

Seeding of the cube as described gives rise to multiple implementation issues. Firstly, as the number of points that are required to be seeded is 8 times larger than the atlas size, the cube dimension that might have been sufficient for seeding only one lattice point per atlas entry may not suffice now. This could be solved by relaxing the requirement of seeding all 8 entries and just seeding the ones that have not yet been seeded. That would, however, cause the atlas entries that are fitted first to have an advantage over the ones at the end of the atlas, which could even result in the latter entries to not be utilized at all. Additionally, we wish to prioritize the seeding of the primary point of each atlas entry, i.e. the lattice point that is the closest. 

Following, we descibe a step-by-step method we use for seeding:
\begin{enumerate}
	\item For each atlas entry, we iterate over its neighbors and select the closest one, which we term as its \emph{primary lattice point}.
	\item We seed the primary lattice points. If a lattice point is primary for two atlas entries, we seed it with only one of them and issue a warning that advises the user to use a higher cube dimension.
	\item For each atlas entry, we iterate over its non-seeded neighbors and seed only the first four, after which we proceed to the next atlas entry. In this way, we attempt to guarantee the representation of each atlas entry in our cube.
	\item For each atlas entry, we seed the remaining non-seeded neighbors.
\end{enumerate}

A better approach, without regard to the time complexity, would to seed only one neighbor per atlas entry per iteration, and repeat as many iterations as it would take to seed all the atlas lattice points. An additional improvement would be to seed in the ascending order with regard to the atlas entry's distance to the voxel's corners.

However, the current algorithm solves the issue with non-utilized atlas entries sufficiently. Even if such an error occurs, it is highly likely to be accompanied with the already mentioned primary lattice point error, i.e. two atlas entries having the same primary lattice point, in which case, we issue a warning. This is because such an error evidently points out the insufficient cube size for the given atlas, either due to the atlas's large size or due to its concentration around certain colors (e.g. the Page 14 from the Munsell Book of Colors as in ukazka).

By seeding the cube, we have appointed coefficients to some of the lattice points. These coefficients reconstruct a spectrum that evaluates to an RGB value which we denote as the \emph{lattice RGB}. The difference between the lattice and the target RGB is therefore equal to the distance between the lattice point and its assigned atlas entry in the cube. It is apparent that the distance may be higher than the defined fitting threshold. In such cases, we must ``improve'' upon the coefficients so that the resulting color difference is as low as possible.

Our problem of improving the coefficients satisfies the definition of the \emph{Non-linear Least Squares} problem~\cite{nonLinearLeastSquares}. Non-linear Least Squares is an unconstrained minimization problem in the following form:
\begin{equation} \label{eq:nonLinearLeastSquares}
	 \underset{x}{\text{minimize}} \hspace{0.5em} f(x) = \sum_{i} f_{i}(x)^{2},
\end{equation}
where $x= \{x_{0}, x_{1}, x_{2}, ... \}$ is a parameter block that we are trying to improve (i.e. our coefficients) and $f_{i}$ are so-called \emph{cost functions}. The definition of cost functions is dependant solely on the current problem. In our case, we primarily require to minimize the difference between the lattice and the target RGB. Our secondary requirement is the shape similarity of the original atlas entry curve and the resulting curve of the atlas lattice point. This gives rise to multiple choices for cost functions, such as using the difference between curves along with the Delta E error, using one or multiple cost functions for the RGB error etc\ldots After implementing some of them and testing their performance, the results of which we provide in ref, we decide on neglecting the shape similarity requirement and using three cost functions, each specifying the absolute difference in one of the three axes of the cube.

To solve an optimization problem defined in such a way, we use the CERES solver.

\subsubsection{CERES solver} \label{ssec:ceresSolver}

As already mentioned in~\cref{sec:upliftingMethods}, the CERES solver is an open-source library for solving large optimization problems such as our Non-linear Least Squares problem. It consists of two parts --- a \emph{modeling API} which provides tools for the construction of optimization problems, allowing us to set parameters such as maximum number of iterations of the optimizer and maximum number of consecutive nonmononotic steps; and a \emph{solver API} that controls the minimization algorithm.

To solve a Non-linear Least Squares problem, the solver requires us to specify only a so-called \emph{residual block}, which is a structure defined by the prior coefficients and the cost functions. During the execution, the solver tries to minimize the values of the cost functions (or \emph{residuals}) in the residual block. The execution is aborted and the current best parameter block returned when the solver achieves either the specified number of iterations or nonmonotonic steps. For more information on the specifics of the CERES solver, we refer the interested reader to its documentation by~\citet{ceresNonLinearLeastSquares}.

There are two main downsides to using the CERES solver. Firstly, the maximum allowed size for a parameter block when using a numeric cost function is 9. This means that we are not able to use more than 9 coefficients for storing a spectrum. This issue could be resolved by adding more residual blocks and combining their results. However, as the runtime of both fitting and rendering with 9 coefficients is already substantially high, we decide not to add such an option as it would most likely be unused.

Another, greater issue is the possibility of CERES getting stuck in local minima and therefore produce unsatisfactory results. This may happen due to the following reasons:
\begin{itemize} \label{ceresDeficiency}
	\item the dimension of the cube is too low, i.e. the prior coefficients are extremely distinct from the ideal coefficients, or
	\item the number of coefficients is too high, i.e. the optimizer fails in improving them as a whole
\end{itemize}
In such cases, the optimizer is not capable of leaving the local minima on its own. We therefore apply a simple heuristics, which consists of only slightly altering the first coefficient (or the first and the second) and running the optimizer again. We chose to alter the first coefficients because they influence the shape of the curve the most.

The need for a heuristic suggests considerable difference between the prior and the resulting coefficients, which implies distinct spectral curves. Such a behavior is undesired, as it may result in strong metameric artifacts. Fortunately, this heuristic is rarely triggered. We examine this in ref, where we analyze the success rate of the fitting and also demonstrate the curve differences by showing both the spectral curves of the atlas entries and of the fitted lattice points.

The sigmoid-based method approaches the starting points problem by selecting the point in the center of the cube and initializing its coefficients to zero. As these values are extremely close to the real values of coefficients, the optimizer does not have a problem with the fitting.

We inspire ourselves by this approach and provide an option for starting in the middle as well. This eliminates the obligation of the user to specify an atlas, i.e. if no atlas is specified, the cube is fitted from the center. However, providing such an option requires us to define a set of prior coefficients for the center point. We determine it by iterating over multiple existing color atlases and searching for spectral curves that roughly evaluate to an RGB of $(0.5, 0.5, 0.5)$. The coefficients of such curves are roughly $\{0.5, 0, 0, 0, ... \}$, i.e. all zeroes except for the first coefficient. We use these prior coefficients for all available moments and cube dimensions.

\subsection{Cube fitting}
Once the starting points are successfully fitted, we can use their coefficients as prior for other lattice points. We proceed similarly to the approach in~\cref{alg:upliftingAlgSigmoid}, where the lattice points are fitted in multiple \emph{fitting rounds}, each round attempting to fit the neighbors of the already fitted points. We provide a more detailed description of the principle behind our fitting algorithm in~\cref{alg:upliftingAlgMoments}.

\begin{algorithm}[t!]
	\caption{Fitting of the cube from starting points}
	\label{alg:upliftingAlgMoments}
	\begin{algorithmic}[1]
		\State $n \gets $ user-defined number of coefficients
		\State $fittingRound \gets$ $0$
		\State $unfittedPoints \gets$ a list of all points in $RGBCube \setminus startingPoints$
		\ForAll{$point \in unfittedPoints$}
		\State $point.fittingDistance = MAX\_DOUBLE$
		\EndFor
		\While {$unfittedPoints$ is not empty}
		\State{$currRoundPts \gets$ points from $unfittedPoints$ that have at least one fitted neighbor}
		\ForAll{$point \in currRoundPts$}
		\ForAll{$fittedNeigbor \in point.neighbors$}
		\State $point.coefs \gets fittedNeighbor.coefs$
		\If{$fittedNeighbor \in atlasLatticePoint$} \label{algStep:conversionBegin}
		\State{$spectrum \gets$ reconstruct spectrum from $fittedNeighbor.coefs$}	
		\State{$fittedNeighbor.coefs \gets$ save $spectrum$ with $n$ coefficients} \label{algStep:conversionEnd}
		\EndIf 
		\State $currDistance \gets $ CERES.Solve($point.coefs$, $costFunctions$)
		\If{$currDistance \leq point.fittingDistance$} \label{algStep:improvementStart}
		\State $point.fittingDistance \gets currDistance$
		\State $point.coefs \gets $ coefficients from solver
		\EndIf
		\If{$currDistance \leq fittingThreshold$}
		\State break
		\EndIf \label{algStep:improvementEnd}
		\EndFor
		\While{$point.fittingDistance > fittingThreshold$} \label{algStep:heuristicsStart}
		\State use heuristics to improve upon the current coefficients
		\State run the solver again
		\State repeat steps \ref{algStep:improvementStart} $-$ \ref{algStep:improvementEnd}
		\If{too many iterations of the while cycle have been performed}
		\State break \label{algStep:heuristicsEnd}
		\EndIf
		\EndWhile
		\If{$point.fittingDistance > fittingThreshold$}
		\State remove $point$ from $unfittedPoints$
		\EndIf
		\If{$point$ has tried the coefficients of all of its neighbors}
		\State remove $point$ from $unfittedPoints$
		\EndIf
		\EndFor	
		\State $fittingRound \gets fittingRound+1$
		\EndWhile
	\end{algorithmic}
\end{algorithm}

Similarly to the cube structure, our algorithm also extends the implementation provided in the sigmoid fitting. Two main features are added --- conversion of coefficient representation (i.e. \emph{coefficient recalculation}) and an improvement heuristic.

\paragraph{Coefficient recalculation}

Up until now, all the uplifting techniques mentioned used the same number of coefficients per lattice point. The reasons for this were both consistency of the representation and the possibility of coefficient interpolation. The latter is especially useful in rendering, as interpolation of coefficients instead of whole spectra substantially increases performance.

We, however, chose to use a different number of coefficients for atlas lattice points and regular points. Although this might sound counter-intuitive, we defend our decision by regarding the performance of the optimizer.

As the time execution of the spectral reconstruction is dependant on the number of coefficients, it is evident that by increasing the number of coefficients, we increase the time requirements of the optimizer. Additonaly, having more coefficients implies more possibilities for the optimizer, which subsequently suggests the need for more iterations. This decreases performance even further.

This does not pose a significant problem when fitting atlas lattice points only, as they usually make up only a small portion of the cube's entries. However, if we were to fit the whole cube with the maximum number of coefficients, we would find a noticeable performance decrease as opposed to fitting with, say, 3 coefficients.

If we therefore wish to preserve representation consistency, we need to either accept the high time execution or give up the precision with which we fit atlas entries. The latter is not a realistic option, as the precise atlas fitting is the ultimate goal of this thesis. The increased time execution is similarly unfeasible, as it takes a lot more time. Furthermore, having more coefficients provides no real benefits to the resulting spectra (see ref). We therefore trade it off and allow the user to insert his own desired number of moments.

The problem we solve in steps~\ref{algStep:conversionBegin} through~\ref{algStep:conversionEnd} of our algorithm is that arising when using atlas lattice points' coefficients as prior to regular points. By supporting two distinct representations of a cube's entry, we simply cannot use the same 9 coefficients as prior to, for example, a point that only requires 4. Luckily, this is easily solved --- we just add a \emph{recomputation} function which reconstructs the spectra of the prior point and subsequently saves it with the new, lower number of coefficients. Obviously, such a conversion causes significant loss of spectral information. However, as we do not need the regular lattice points' coefficients to evaluate to any specific spectra, this does not need to concern us.

Up until now, we have not mentioned the effect of various numbers of coefficients on the interpolation phase of rendering, specifically on the coefficient interpolation. That is because our implementation, similarly to the sigmoid implementation, does not support this and is rather based on the interpolation of spectra. We explain the reasoning behind this in ref DOROBIT. However, if required, adding support for coefficient interpolation does not pose a problem --- we would only need to utilize the already existing recomputation function. We would need to proceed in the opposite manner as before, when we converted 9-coefficient representations to a smaller number of coefficients. Instead, we would need to convert the regular lattice points to their 9-coefficient representation, so as to preserve the spectral precision of the atlas lattice points.

\paragraph{Improvement heuristic}

To minimize the shortcomings of the optimizer mentioned in~\cref{ceresDeficiency}, we add a heuristic-based improvement of the coefficients, implemented in steps~\ref{algStep:heuristicsStart} through~\ref{algStep:heuristicsEnd}. Its implementation consists of solely slightly changing the first coefficient and running the optimizer again for a pre-defined number of times. In constrast to the heuristics applied when fitting atlas entries, we ommit the changing of the second coefficient completely and we also use significantly less iterations when changing the first coefficient. This is because we do not necessarily require the points to be fitted in the given round. Even if the fitting fails, it may still be successful in the following rounds, where the coefficients of newly fitted neighbors might be used as prior. Additionaly, after the cube fitting is done, we still run an ``improvement'' step for the unsuccessful points. We discuss the specifics of this step in~\cref{ssec:cubeImprovement}.

The number of the fitting rounds depends both on the cube dimension and on the kind of atlas that has been used. If we choose not to input an atlas, the fitted cube ``grows'' from the middle, while seeding with an atlas makes the cube ``grow'' from many places, requiring a lot less round. We show the differences in , where we present the 

Obviously, neither option is better in terms of performance, as the number of fitted points must still be the same. 

\subsection{Cube improvement} \label{ssec:cubeImprovement}

In extreme cases, such as when using too many coefficients to store entries of a cube with either a low dimension or a low fitting threshold, fitting some of the points may be unsuccessful even after applying heuristic improvements. We attempt to solve this by simply iterating over all the already fitted points and using their coefficients as prior for our point. We even apply heuristics similar to the ones used during the fitting of atlas entries. This shows to be useful especially if we have an extremely limited number of sets of prior coefficients, which happens usually if the overall size of the cube is small.

We call this process \emph{cube improvement}. Although the heuristic may seem simple and illogical at first, we must once again zvyraznit that it is aimed solely at the most extreme cases. As a matter of fact, it usually does not even get triggered and, if it does, it is only for a highly problematic point of two. We present the exact statistics in ref.

Obviosuly, improving the points as such has its drawbacks. We do not focus on the time performance, as we mainly want this part to be successful rather than fast. Moreover, as this step runs for a small set of numbers only, the execution time is negligible in comparison to other steps.

The more important issue is the shape of the resulting curve. Using seemingly ``random'' coefficients as prior suggests the resulting curve's shape to be unlike those of its neighbors, which may result in metameric artifacts.


The reason behind using improvement heuristics both during and after fitting is both for improved time performance, i.e. a simple change of coefficients is less time-consuming that the process t

and a more desirable resulting shape of the spectral curves. Cube improvement after fitting is extremely time-consuming in contrast to a simple change in coefficients. Moreover, due to reasons explained in~\cref{ssec:cubeImprovement}, the resulting curve could take a shape dramatically distinct from the shape of its neighbors, which is an undesirable behavior. On the other hand, trying to improve all coefficients during the cube fitting would require a complicated heuristic including a lot of optimizer runs, which might eventually lead to a lowered time performance. We show all of these effects in ref.


multithreaded 

We support from 2-8 moments then, however, we will talk about the number of coefficients (although user inserts moments). From now on we talk about coefficients. We do not support 1 or 2 coefs as they only create rovne ciary which does not reconstruct the color we want.

\subsection{Cube storage}
	
Once the cube is completely fitted, its contents are written to a binary file. As we want the resulting file to be as small as possible, we save only the information crucial for the purposes of rendering. Following, we provide a list of contents of a cube file:
\begin{itemize}
	\item \emph{version}
	\item \emph{features}, i.e. whether the cube contains debug information
	\item \emph{moment flag} --- a flag signifying that the cube is based on trigonometric moments. We extend the sigmoid cube structure with a sigmoid flag for an easier cube recognition in rendering software.
	\item \emph{dimension}
	\item \emph{illuminant} under which the cube has been fitted
	\item \emph{fitting threshold}
	\item for every point, we store:
	\begin{itemize}
		\item \emph{coefficient count}, which gives us information on whether the point is an atlas lattice point or a regular point
		\item \emph{coefficients}
		\item \emph{lattice RGB}
		\item \emph{fitting distance}, i.e. the distance between lattice and target RGB, should be below fitting threshold
		\item in case the debug information is included, the entry also contains:
		\begin{itemize}
			\item \emph{target RGB}
			\item \emph{treated variable}, i.e. in which round was the point fitted. If the variable is equal to $-1$, it means the fitting has failed for this point.
		\end{itemize}
	\end{itemize}
\end{itemize}

If we were to use such a cube for the purposes of rendering, the knowledge of both the entry's treated paramter and its target RGB is unnecessary. The only benefit the treated parameter provides is the capability to issue a warning in case the cube is incomplete. This can, however, also be achieved by spectral reconstruction from the coefficients and the comparison of its RGB to that of the entry's lattice point.

Storing the target RGB variable also provides no irreplaceable advantages. The target RGB of all the lattice points can simply be computed from the cube's dimension parameter, rendering such variables useless.

In addition to the storing of the cube, we also provide a function capable of loading such a cube and initializing its parameters. Borgtool utilizes this function in case the user wants to create a texture from an already existing cube or simply wants to check the correctness of the existing cube's parameters. The code of this function is also used during the integration in a rendering software.

\section{ART integration}

The user can choose the size of the cube etc. Borgtool has the following options, and we implement all of them for our method (except for something)
The borgtool already has the sigmoid method implemented as in algorithm daco
Uses three coefficients, fits from middle, and therefore results are metameric (show picture)
We implement the trigonometric moment method by changing up the code provided by peters
We aim for allowing the user to add a color atlas, however we also provide an option when the user fits from the middle
When fitted from middle, the optimizer works very similarly to the sigmoid method (show pictures)

Also explain the threshold value - it is really important to set it properly and that it affects performance.


 as the interpolation of multiple spiky, non-similar spectra often results in a similarly uneven spectrum susceptible to metameric artifacts. Mention here that we NEED to have smooth spectra and that it is very important for interpolation. We therefore set parameters so that this is possible as this is the key. 
